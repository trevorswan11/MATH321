\documentclass[12pt]{article}

% Custom Commands
\newcommand{\set}[1]{\left\{ {#1} \right\}}
\newcommand{\limtoinf}[1][n]{\lim_{ {#1} \to \infty}}
\newcommand{\liminftoinf}[1][n]{\liminf_{ {#1} \to \infty}}
\newcommand{\limsuptoinf}[1][n]{\limsup_{ {#1} \to \infty}}
\newcommand{\abs}[1]{\left| {#1} \right|}
\newcommand{\ceil}[1]{\lceil {#1} \rceil}
\newcommand{\floor}[1]{\lfloor {#1} \rfloor}
\newcommand{\seq}[2][n]{\left\{ {#2} \right\}_{#1=1}^\infty}
\newcommand{\paren}[1]{\left( {#1} \right)}
\newcommand{\bR}{\mathbb{R}}
\newcommand{\bZ}{\mathbb{Z}}
\newcommand{\bN}{\mathbb{N}}
\newcommand{\bQ}{\mathbb{Q}}

% Math and symbol packages
\usepackage{amsmath}
\usepackage{amsthm}
\usepackage{amssymb}
\usepackage{mathtools}
\usepackage{derivative}

% Formatting
\usepackage{inputenc}
\usepackage[left=2.54cm,right=2.54cm,top=2.54cm,bottom=2.54cm]{geometry}
\usepackage{fancyhdr}
\usepackage{lipsum}

% Actual content
\begin{document}
\pagestyle{fancy}
\setlength{\headheight}{14.49998pt}
\fancyhead[L]{Trevor Swan}
\fancyhead[C]{\textbf{MATH321 - HW5}}
\fancyhead[R]{10/04/24}
\fancyfoot[C]{\thepage}

% Begin Problem 1
\noindent \textbf{(2.3.7)} Let $\seq{x_n}$ and $\seq{y_n}$ be bounded sequences.

(a) Show that $\seq{x_n+y_n}$ is bounded.

\begin{proof}
	\lipsum[1]
\end{proof}

(b) Show that

\begin{align*}
	\paren{\liminftoinf x_n}+\paren{\liminftoinf y_n}\le\liminftoinf\paren{x_n+y_n}
\end{align*}

\begin{proof}
	\lipsum[1]
\end{proof}

(c) Find an explicit $\seq{x_n}$ and $\seq{y_n}$ such that

\begin{align*}
	\paren{\liminftoinf x_n}+\paren{\liminftoinf y_n}<\liminftoinf\paren{x_n+y_n}
\end{align*}

\begin{proof}
	\lipsum[1]
\end{proof}

\newpage

% Begin Problem 2
\noindent \textbf{(2.3.8)}  Let $\seq{x_n}$ and $\seq{y_n}$ be bounded sequences, and $\seq{x_n+y_n}$ is bounded by the previous exercise.

(a) Show that

\begin{align*}
	\paren{\limsuptoinf x_n}+\paren{\limsuptoinf y_n}\ge\limsuptoinf\paren{x_n+y_n}
\end{align*}

\begin{proof}
	\lipsum[1]
\end{proof}

(b) Find an explicit $\seq{x_n}$ and $\seq{y_n}$ such that

\begin{align*}
	\paren{\limsuptoinf x_n}+\paren{\limsuptoinf y_n}>\limsuptoinf\paren{x_n+y_n}
\end{align*}

\begin{proof}
	\lipsum[1]
\end{proof}

\newpage

% Begin Problem 3
\noindent \textbf{(2.4.1)} prove that $\seq{\frac{n^2-1}{n^2}}$ is Cauchy using directly the definition of Cauchy sequences.

\begin{proof}
	\lipsum[1]
\end{proof}

\newpage

% Begin Problem 4
\noindent \textbf{(2.4.6)} Suppose $\abs{x_n-x_k}\le\frac{n}{k^2}$ for all $n$ and $k$. Show that $\seq{x_n}$ is Cauchy.

\begin{proof}
	\lipsum[1]
\end{proof}

\newpage

% Begin Problem 5
\noindent \textbf{(2.4.8)} True or false, prove or find a counterexample: If $\seq{x_n}$ is a Cauchy sequence, then there exists an $M$ such that for all $n\ge M$, we have $\abs{x_{n+1}-x_n}\le\abs{x_n-x_{n-1}}$.

\begin{proof}
	\lipsum[1]
\end{proof}

\end{document}
