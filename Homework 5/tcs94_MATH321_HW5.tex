\documentclass[12pt]{article}

% Custom Commands
\newcommand{\set}[1]{\left\{ {#1} \right\}}
\newcommand{\limtoinf}[1][n]{\lim_{ {#1} \to \infty}}
\newcommand{\liminftoinf}[1][n]{\liminf_{ {#1} \to \infty}}
\newcommand{\limsuptoinf}[1][n]{\limsup_{ {#1} \to \infty}}
\newcommand{\abs}[1]{\left| {#1} \right|}
\newcommand{\ceil}[1]{\lceil {#1} \rceil}
\newcommand{\floor}[1]{\lfloor {#1} \rfloor}
\newcommand{\seq}[2][n]{\left\{ {#2} \right\}_{#1=1}^\infty}
\newcommand{\paren}[1]{\left( {#1} \right)}
\newcommand{\bR}{\mathbb{R}}
\newcommand{\bZ}{\mathbb{Z}}
\newcommand{\bN}{\mathbb{N}}
\newcommand{\bQ}{\mathbb{Q}}

% Math and symbol packages
\usepackage{amsmath}
\usepackage{amsthm}
\usepackage{amssymb}
\usepackage{mathtools}
\usepackage{derivative}

% Formatting
\usepackage{inputenc}
\usepackage[left=2.54cm,right=2.54cm,top=2.54cm,bottom=2.54cm]{geometry}
\usepackage{fancyhdr}
\usepackage{lipsum}

% Spacing formats
\AtBeginDocument{
	\setlength{\abovedisplayskip}{-6pt}
	\setlength{\belowdisplayskip}{5pt}
}

% Actual content
\begin{document}
\pagestyle{fancy}
\setlength{\headheight}{14.49998pt}
\fancyhead[L]{Trevor Swan}
\fancyhead[C]{\textbf{MATH321 - HW5}}
\fancyhead[R]{10/04/24}
\fancyfoot[C]{\thepage}

% Begin Problem 1
\noindent \textbf{(2.3.7)} Let $\seq{x_n}$ and $\seq{y_n}$ be bounded sequences.

(a) Show that $\seq{x_n+y_n}$ is bounded.

\begin{proof}
	Suppose $M_1,M_2\in\bR$. Since $\seq{x_n}$ is bounded, the there exists some $M_1>0$ such that for all $n\in\bN, \abs{x_n}\le M_1$. Similarly since $\seq{y_n}$ is bounded, there exists some $M_2>0$ such that for all $n\in\bN, \abs{y_n}\le M_2$. 
	
	\indent Now consider the sequence $\seq{x_n + y_n}$. By the triangle inequality, $\abs{x_n+y_n}\le\abs{x_n}+\abs{y_n}$ holds. We can then say that $\abs{x_n+y_n}\le\abs{x_n}+\abs{y_n}\le M_1+M_2$. Defining $M=M_1+M_2$, this is equivalent to $\abs{x_n+y_n}\le M$. As $M_1$ and $M_2$ are arbitrary positive real numbers, $M$ is also a positive real number. Thus, by definition the sequence $\seq{x_n+y_n}$ is bounded, as required.
\end{proof}

(b) Show that

\begin{align*}
	\paren{\liminftoinf x_n}+\paren{\liminftoinf y_n}\le\liminftoinf\paren{x_n+y_n}
\end{align*}

\begin{proof}
	Let $\seq[k]{x_{n_k}}$ be a subsequence of $\seq{x_n}$. Since $\seq{x_n}$ is bounded, then $\seq[k]{x_{n_k}}$ must also be bounded. We can then choose a convergent subsequence $\seq[i]{x_{n_{k_i}}}$ of $\seq[k]{x_{n_k}}$, which is guaranteed to exist by the Bolzano-Weierstrass Theorem. Therefore $\limtoinf[i] \paren{x_{n_{k_i}}}$ exists and $\liminftoinf x_n\le\limtoinf[i] \paren{x_{n_{k_i}}}$ by the definition of the limit inferior.
	
\indent By similar reasoning we can let $\seq[k]{y_{n_k}}$ be a subsequence of $\seq{y_n}$ and can chose a subsequence $\seq[i]{y_{n_{k_i}}}$ such that $\limtoinf[i] \paren{y_{n_{k_i}}}$ exists. Hence $\liminftoinf y_n\le\limtoinf[i] \paren{y_{n_{k_i}}}$ by the definition of the limit inferior.

\indent As you'd expect, we can also choose a subsequence $\seq[i]{x_{n_{k_i}}+y_{n_{k_i}}}$ of $\seq[k]{x_{n_{k}}+y_{n_{k}}}$ which is a subsequence of $\seq{x_n+y_n}$. By the Bolzano-Weierstrass Theorem, we can choose this subsequence such that it converges and thus $\limtoinf[i]\paren{x_{n_{k_i}}+y_{n_{k_i}}}=\limtoinf[k]\paren{x_{n_k}}+\limtoinf[k]\paren{x_{n_k}}$ and that $\liminftoinf\paren{x_n+y_n}\ge\limtoinf[i]\paren{x_{n_{k_i}}+y_{n_{k_i}}}$ by definition of the limit inferior. Using the previously derived expressions, we have

\begin{align*}
	\limtoinf[i]\paren{x_{n_{k_i}}+y_{n_{k_i}}}=&\limtoinf[i]\paren{x_{n_{k_i}}} + \limtoinf[i]\paren{y_{n_{k_i}}} \\
	\limtoinf[i]\paren{x_{n_{k_i}}+y_{n_{k_i}}}\ge&\liminftoinf x_n +\liminftoinf y_n \\
	\liminftoinf\paren{x_n+y_n}\ge\limtoinf[i]\paren{x_{n_{k_i}}+y_{n_{k_i}}}\ge&\liminftoinf x_n +\liminftoinf y_n
\end{align*}

\noindent Therefore $\paren{\liminftoinf x_n}+\paren{\liminftoinf y_n}\le\liminftoinf\paren{x_n+y_n}$ as required.

\end{proof}

\newpage

(c) Find an explicit $\seq{x_n}$ and $\seq{y_n}$ such that

\begin{align*}
	\paren{\liminftoinf x_n}+\paren{\liminftoinf y_n}<\liminftoinf\paren{x_n+y_n}
\end{align*}

\begin{proof}
	Let $x_n=(-1)^n$ and $y_n=(-1)^{n+1}$ for all $n\in\bN$. Clearly both of these sequences are bounded above by 1 and below by -1.
	
\begin{align*}
	x_n+y_n=(-1)^n+(-1)^{n+1}=(-1)^n+(-1)^n(-1)=(-1)^n(1-1)=0
\end{align*}

Thus $\liminftoinf\paren{x_n+y_n}=0$. The minimum value for $x_n, y_n$ is -1 for all $n\in\bN$, so $\liminftoinf x_n=-1$ and $\liminftoinf y_n=-1$. Therefore $\liminftoinf x_n+\liminftoinf y_n=-1+-1=-2$. Consequently,

\begin{align*}
	\paren{\liminftoinf x_n}+\paren{\liminftoinf y_n}<&\liminftoinf\paren{x_n+y_n} \\
	-2<0
\end{align*}

\noindent showing that the strict inequality (no equality) holds for these choices of $x_n,y_n$.
\end{proof}

\newpage

% Begin Problem 2
\noindent \textbf{(2.3.8)}  Let $\seq{x_n}$ and $\seq{y_n}$ be bounded sequences, and $\seq{x_n+y_n}$ is bounded by the previous exercise.

(a) Show that

\begin{align*}
	\paren{\limsuptoinf x_n}+\paren{\limsuptoinf y_n}\ge\limsuptoinf\paren{x_n+y_n}
\end{align*}

\begin{proof}
	Choose convergent subsequences $\seq[i]{x_{n_{k_i}}}, \seq[i]{x_{n_{k_i}}}$ of $\seq{x_{n_k}}$ and $\seq{y_{n_k}}$, respectively as done in the previous exercise. Therefore the limits and limit superiors can be related as $\limsuptoinf{x_n}\ge\limtoinf[i]{x_{n_{k_i}}}$ and $\limsuptoinf{y_n}\ge\limtoinf[i]{y_{n_{k_i}}}$ by the definition of the limit superior. Note that these limits are guaranteed to exist as the subsequences were chosen to converge and $\seq{x_n}$ and $\seq{y_n}$ are given to be bounded.
	
\indent Similarly, we can also choose a subsequence $\seq[i]{x_{n_{k_i}}+y_{n_{k_i}}}$ of $\seq[k]{x_{n_{k}}+y_{n_{k}}}$. By the Bolzano-Weierstrass Theorem, we can chose this subsequence to converge. In other words, $\limtoinf[i]\paren{x_{n_{k_i}}+y_{n_{k_i}}}=\limtoinf[k]\paren{x_{n_k}}+\limtoinf[k]\paren{x_{n_k}}$. Also similarly to the above proof, $\limsuptoinf\paren{x_n+y_n}\le\limtoinf[i]\paren{x_{n_{k_i}}+y_{n_{k_i}}}$ by definition of the limit superior. Consequently we have that,

\begin{align*}
	\limtoinf[i]\paren{x_{n_{k_i}}+y_{n_{k_i}}}=&\limtoinf[i]\paren{x_{n_{k_i}}} + \limtoinf[i]\paren{y_{n_{k_i}}} \\
	\limtoinf[i]\paren{x_{n_{k_i}}+y_{n_{k_i}}}\le&\limsuptoinf x_n +\limsuptoinf y_n \\
	\limsuptoinf\paren{x_n+y_n}\le\limtoinf[i]\paren{x_{n_{k_i}}+y_{n_{k_i}}}\le&\limsuptoinf x_n +\limsuptoinf y_n
\end{align*}

\noindent Therefore, $\paren{\limsuptoinf x_n}+\paren{\limsuptoinf y_n}\ge\limsuptoinf\paren{x_n+y_n}$ as required.
\end{proof}

(b) Find an explicit $\seq{x_n}$ and $\seq{y_n}$ such that

\begin{align*}
	\paren{\limsuptoinf x_n}+\paren{\limsuptoinf y_n}>\limsuptoinf\paren{x_n+y_n}
\end{align*}

\begin{proof}
	Let $x_n=(-1)^n$ and $y_n=(-1)^{n+1}$ for all $n\in\bN$. Clearly both of these sequences are bounded above by 1 and below by -1.
	
\begin{align*}
	x_n+y_n=(-1)^n+(-1)^{n+1}=(-1)^n+(-1)^n(-1)=(-1)^n(1-1)=0
\end{align*}

Thus $\limsuptoinf\paren{x_n+y_n}=0$. The maximum value for $x_n, y_n$ is 1 for all $n\in\bN$, so $\limsuptoinf x_n=1$ and $\limsuptoinf y_n=1$. Therefore $\limsuptoinf x_n+\limsuptoinf y_n=1+1=2$. Consequently,

\begin{align*}
	\paren{\limsuptoinf x_n}+\paren{\limsuptoinf y_n}>&\limsuptoinf\paren{x_n+y_n} \\
	2>0
\end{align*}

\noindent showing that the strict inequality (no equality) holds for these choices of $x_n,y_n$.
\end{proof}

\newpage

% Begin Problem 3
\noindent \textbf{(2.4.1)} Prove that $\seq{\frac{n^2-1}{n^2}}$ is Cauchy using directly the definition of Cauchy sequences.

\begin{proof}
	\lipsum[1]
\end{proof}

\newpage

% Begin Problem 4
\noindent \textbf{(2.4.6)} Suppose $\abs{x_n-x_k}\le\frac{n}{k^2}$ for all $n$ and $k$. Show that $\seq{x_n}$ is Cauchy.

\begin{proof}
	\lipsum[1]
\end{proof}

\newpage

% Begin Problem 5
\noindent \textbf{(2.4.8)} True or false, prove or find a counterexample: If $\seq{x_n}$ is a Cauchy sequence, then there exists an $M$ such that for all $n\ge M$, we have $\abs{x_{n+1}-x_n}\le\abs{x_n-x_{n-1}}$.

\begin{proof}
	\lipsum[1]
\end{proof}

\end{document}
