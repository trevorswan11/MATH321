\documentclass[12pt]{article}
\newcommand{\set}[1]{\left\{ {#1} \right\}}
\newcommand{\bR}{{\mathbb{R}}}
\newcommand{\bZ}{{\mathbb{Z}}}
\newcommand{\bN}{{\mathbb{N}}}
\newcommand{\bQ}{{\mathbb{Q}}}

% Math and symbol packages
\usepackage{amsmath}
\usepackage{amsthm}
\usepackage{amssymb}
\usepackage{mathtools}
\usepackage{derivative}

% Formatting
\usepackage{inputenc}
\usepackage[left=2.54cm,right=2.54cm,top=2.54cm,bottom=2.54cm]{geometry}
\usepackage{fancyhdr}
\usepackage{lipsum}

% Actual content
\begin{document}
\pagestyle{fancy}
\setlength{\headheight}{14.49998pt}
\fancyhead[L]{Trevor Swan}
\fancyhead[C]{\textbf{MATH321 - HW2}}
\fancyhead[R]{09/11/24}
\fancyfoot[C]{\thepage}

% Start Problem 1
\noindent \textbf{(1.2.8)} Show that for every pair of real numbers $x$ and $y$ such that $x < y$, there exists an irrational number $s$ such that $x < s < y$.

\begin{proof}
	The set of the rational numbers $\bQ$ is known to be dense in the real numbers $\bR$. This means that given $x,y\in\bR$ and $x<y$, then there exists some rational number $r\in\bQ$ such that $x<r<y$. Note that this is part of the Archimedean property. \newline
	
\indent Consider the two real numbers $\frac{x}{\sqrt{2}}$ and $\frac{y}{\sqrt{2}}$. Building off the earlier proofs in this course, it is known that $\sqrt{2}$ is irrational, and that the irrational numbers are a subset of $\bR$. It is also known that $x<y$, so it follows that $\frac{x}{\sqrt{2}} < \frac{y}{\sqrt{2}}$ as $\bR$ is an ordered field. This is settled on Page 26 of \textit{Basic Analysis I: Introduction to Real Analysis, Volume 1.} by Lebl, Jiri. By the density of the rationals in the reals, there exists some $r\in\bQ$ such that $\frac{x}{\sqrt{2}} < r < \frac{y}{\sqrt{2}}$. Now, multiply the expression by $\sqrt{2}$, preserving the ordering though showing a different property. This yields the expression $x < \sqrt{2}r < y$. Both $x$ and $y$ are real numbers, and while $\sqrt{2}r$ is also a real number, it must be proved that it is irrational.

\indent Again, it is known that $r$ is a rational number and $\sqrt{2}$ is an irrational number. Assume to the contrary that $\sqrt{2}r$ is a rational number $q\in\bQ$. In other words, $\sqrt{2}r = q$, where $r,q\in\bQ$. Since $r$ and $q$ are both rational numbers, they can be expressed as $r=\frac{a}{b}$ and $q=\frac{c}{d}$, where $b, q\neq0$ and $a,b,c,d\in\bZ$. Using these ratios, solve for $\sqrt{2}$ from the expression $\sqrt{2}r = q$ as

\begin{equation*}
	\sqrt{2}=\frac{q}{r}=\frac{\frac{c}{d}}{\frac{a}{b}}=\frac{bc}{ad}
\end{equation*}

\noindent where $r$ and $q$ are substituted accordingly.

\indent Note that this means $\sqrt{2}=\frac{bc}{ad}$, in which a contradiction is arrived at. Since $a,b,c,d\in\bZ$ and by extension $cd, ab\in\bZ$, then $\sqrt{2}$ must be rational as it can be expressed as a ratio of integers. This contradicts the assumption that $\sqrt{2}$ is irrational. Therefore $\sqrt{2}r$ is irrational.

\indent Choosing $s=\sqrt{2}r$, it is shown that there exists some irrational number $s$ between real numbers $x$ and $y$ given that $a<y$.
\end{proof} 

\newpage

% Start Problem 2
\noindent \textbf{(1.2.9)} Let $A$ and $B$ be two nonempty bounded sets of real numbers. Let \\ $C \coloneq \set{a+b : a\in A, b\in B}$. Show that $C$ is a bounded set and that

\begin{align*}
	\sup C = \sup A + \sup B \quad \text{and} \quad \inf C = \inf A + \inf B.
\end{align*}

\begin{proof}
	Since $A$ and $B$ are both bounded and nonempty subsets of the real numbers, then they must have finite bounds which exist. Hence, define the respective bounds as 
	
\begin{align*}
	\inf A = n_a, \sup A = N_A \quad\text{and}\quad \inf B = n_B, \sup B = N_B
\end{align*}

\noindent In other words, for all $a\in A$ and $b\in B$, $n_a \le a \le N_A$ and $n_B \le b \le N_B$.

\indent For any $c\in C$, there exists an $a\in A$ and $b\in B$ such that $c=a+b$ as given. Since $a \le N_A$ and $b \le N_B$, these inequalities can be added to yield $a + b \le N_A + N_B$. Therefore $c \le N_A + N_B$ and hence $N_A + N_B$ is an upper bound for c. Similarly, since $a \ge n_B$ and $b \ge n_B$, these inequalities can also be added to yield  $a + b \ge n_A + n_B$. Therefore $c \ge n_A + n_B$ and hence $n_A + n_B$ is a lower bound for $C$. Since $C$ has both an upper bound $N_A + N_B$ and a lower bound $n_A + n_B$, it follows that $C$ is a bounded set.

\indent For every $a\in A$ and $b\in B$, $c=a+b$. Since, by the definition of the supremum, $a\le \sup A$ for all $a\in A$ and $b \le \sup B$ for all $b\in B$, the inequalities can be combined to yield $a+b\le \sup A + \sup B$. Hence, every element of $C$ is less than or equal to $\sup A + \sup B$, and it follows that $\sup C \le \sup A + \sup B$.

\indent Now, choose an element $a'\in A$ who's arbitrarily close to $\sup A$ and similarly choose an element $b'\in B$ who's arbitrarily close to $\sup B$. Consider the pair $a', b'$, which yields the sum $c'=a'+b'$. This sum $c'$ will be arbitrarily close to $\sup A + \sup B$. Therefore, it follows that the least upper bound of $C, \sup C$, must be at least $\sup A+\sup B$. Thus, $\sup C \ge \sup A + \sup B$. Consequently $\sup C = \sup A + \sup B$, as both $\sup C \le \sup A + \sup B$ and $\sup C \ge \sup A + \sup B$.

\indent For any $a\in A$ and $b\in B$, $c=a+b$. Since $a\ge \inf A$ for all $a\in A$ and $b\ge\inf B$ for all $b\in B$ by definition, which can be combined to yield $a + b \ge \inf A + \inf B$. Hence, every element of $C$ is greater than or equal to $\inf A + \inf B$, and it follows that $\inf C \ge \inf A + \inf B$.

\indent Similarly to the argument above, choose an element $a'\in A$ who's arbitrarily close to $\inf A$ and similarly choose an element $b'\in B$ who's arbitrarily close to $\inf B$. Consider the pair $a', b'$, which yields the sum $c'=a'+b'$. This sum $c'$ will be arbitrarily close to $\inf A + \inf B$. Therefore, it follows that the greatest lower bound of $C, \inf C$, must be at most $\inf A + \inf B$. Thus, $\inf C \le \inf A + \inf B$. Consequently $\inf C = \inf A + \inf B$, as both $\inf C \ge \inf A + \inf B$ and $\inf C \le \inf A + \inf B$.

\indent In summary, when given two nonempty bounded sets of real numbers, $A$ and $B$ with $C \coloneq \set{a+b : a\in A, b\in B}$, it follows that $C$ is bounded and that

\begin{align*}
	\sup C = \sup A + \sup B \quad \text{and} \quad \inf C = \inf A + \inf B.
\end{align*}

\end{proof}

\newpage

% Start Problem 3
\noindent \textbf{(1.2.12)} Prove \underline{Proposition 1.2.8}. If $S \subset \bR$ is nonempty and bounded above, then for every $\epsilon > 0$ there exists an $x\in S$ such that $(\sup S)-\epsilon < x \le \sup S$.

\begin{proof}
	Let $M\coloneq\sup S$ be the supremum of $S$. By definition of the supremum, $M$ is the smallest number such that $M$ is an upper bound of $S$ where for all $x \in S$, it holds that $x\le M$.
	\indent Assume to the contrary that no such $x\in S$ exists to satisfy $(\sup S)-\epsilon < x \le \sup S$ for any $\epsilon > 0$. This means that for every $x\in S$, it holds that $x\le M -\epsilon$. Consequently, $M-\epsilon$ would be an upper bound for $S$ as $x\le M-\epsilon$ for all $x\in S$. Thus, $M-\epsilon$ must be an upper bound. However, $M$ is already assumed to be the supremum of $S$, meaning that $M$ is the least upper bound. $M-\epsilon$ being an upper bound contradicts the fact that $M$ is the least upper bound for the set $S$ as $M-\epsilon\le M$. Hence, the assumption that no such $x\in S$ exists to satisfy $(\sup S)-\epsilon < x \le \sup S$ for any $\epsilon > 0$ is false.
	
\end{proof}

\newpage

% Start Problem 4
\noindent \textbf{(1.3.5)} Let $f : D\to \bR$ and $g : D\to \bR$ be function with D being nonempty. \newline

\noindent (a) Suppose $f(x)\le g(y)$ for all $x\in D$ and $y\in D$. Show that

\begin{equation*}
	\underset{x\in D}{\sup} f(x) \le \underset{x\in D}{\inf} g(x).
\end{equation*} 

\begin{proof}
	As argued on Page 38 of \textit{Basic Analysis I: Introduction to Real Analysis, Volume 1.} by Lebl, Jiri, the $x$'s on either side of the inequality are not necessarily the same. For ease of the argument, refer to the function $g$ as a function of $y$, which is acceptable because it is given that $y\in D$. 
	
\indent Let $M=\underset{x\in D}{\sup} f(x)$ and $m=\underset{x\in D}{\inf} g(x)$. By the definition of the supremum, $M$ is the least upper bound for $f(x)$ for all $x\in D$. This means that $M\ge f(x)$ for all $x\in D$. On the other hand, $m$ is, by definition, the greatest lower bound for $g(y)$ for all $y\in D$. This means that $m\le g(y)$ for all $y\in D$.

\indent It is given that $f(x)\le g(y)$ for all $x\in D$ and $y\in D$. Also, $m\le g(y)$ as shown above. These two inequalities can be combined based on the given as $f(x)\le m$ for all $x,y\in D$. Since $f(x)\le m$ for all $x\in D$, $m$ is an upper bound for $f$. Recall that $M$, the supremum of $f$, is the least upper bound for $f$, so it must be true that $M\le m$. Re-substituting the appropriate values for $M$ and $m$ yields the expression 

\begin{equation*}
	\underset{x\in D}{\sup} f(x) \le \underset{x\in D}{\inf} g(x).
\end{equation*}

\end{proof}

\newpage

\noindent (b) Find a specific $D, f, \text{ and } g$, such that $f(x) \le g(x)$ for all $x\in D$, but

\begin{equation*}
	\underset{x\in D}{\sup} f(x) > \underset{x\in D}{\inf} g(x).
\end{equation*}

\begin{proof}
	Choose $D\coloneq\bN$, and let $f(x)=\frac{1}{x}$, $g(x)=x$ for $x=[1,10], x\in D$. The supremum of $f(x)$. It is known that $f(x)$ is a decreasing function, with $f(1)=\frac{1}{1}=1$ and $f(10)=\frac{1}{10}=0.1$. Therefore, $f(x)$ achieves its maximum value at $f(1)$. In other words, the supremum of $f(x)$ is $1$, written as $\underset{x\in D}{\sup} f(x)=1$. For $g(x)$, it is known to be a constantly increasing function, with $g(1)=1$ and $g(10)=10$. Therefore, $g(x)$ achieves its minimum value at $g(1)$. In other words, the infimum of $g(x)$ is $1$, written as $\underset{x\in D}{\inf} g(x) = 1$. Because $g(x)$ achieves its minimum at $x=1$, and $f(x)$ achieves its maximum at that value as well, the inequality $f(x) \le g(x)$ holds true for all $x\in D$. Putting these facts together
	
\begin{align*}
	\underset{x\in D}{\sup} f(x) = 1 \quad \text{and} \quad \underset{x\in D}{\inf} g(x) = 1
\end{align*}

\noindent Therefore the strict inequality, with values plugged in, $\underset{x\in D}{\sup} f(x) > \underset{x\in D}{\inf} g(x)$ cannot be true as $\underset{x\in D}{\sup} f(x)=\underset{x\in D}{\inf} g(x)=1$, and it is incorrect to say $1>1$, without showing equality.

Suppose then that $g(x)=x+1$, and $f(x)$ along with $D$ remain untouched. Similar logic can be followed, yielding $\underset{x\in D}{\inf} g(x) = 2$. As $f(x)$ was untouched, $\underset{x\in D}{\sup} f(x) = 1$. Therefore the inequality $\underset{x\in D}{\sup} f(x) > \underset{x\in D}{\inf} g(x)$ is untrue even when the strict inequality $f(x) < g(x)$ is true.

\end{proof}

\newpage

% Start Problem 5
\noindent \textbf{(1.3.7)} Let $D$ be a nonempty set. Suppose $f : D\to\bR$ and $g : D\to \bR$ are bounded functions. \newline

\noindent (a) Show

\begin{align*}
	\underset{x\in D}{\sup}(f(x) + g(x))\le\underset{x\in D}{\sup}f(x) + \underset{x\in D}{\sup}g(x) \quad \text{and} \quad \underset{x\in D}{\inf}(f(x)+g(x))\ge \underset{x\in D}{\inf}f(x) + \underset{x\in D}{\inf}g(x)
\end{align*}

\begin{proof}
	Let $M=\underset{x\in D}{\sup}f(x)$ and $N=\underset{x\in D}{\sup}g(x)$. By the definition of the supremum, $f(x)\le M$ and $g(x)\le N$ for all $x\in D$. These two inequalities can be added to show that $f(x)+g(x)\le M + N$ for all $x\in D$. This means that all pairs of $f$ and $g$ for all $x\in D$, their sum is less than or equal to the sum of the individual function's supremum. It is known that the functions sum is bounded, as both $f(x)$ and $g(x)$ are given to be bounded. Consequently, the least upper bound of $f(x)+g(x)$ must also be less than or equal to $M+N$. By definition, the least upper bound of $f(x)+g(x) \coloneq \underset{x\in D}{\sup}(f(x) + g(x))$. Substituting this expression into $f(x)+g(x)\le M + N$, along with known values of $M$ and $N$, it can be seen that
	
\begin{align*}
	\underset{x\in D}{\sup}(f(x) + g(x))\le\underset{x\in D}{\sup}f(x) + \underset{x\in D}{\sup}g(x)
\end{align*}

Now, let $m=\underset{x\in D}{\inf}f(x)$ and $n=\underset{x\in D}{\inf}g(x)$. By the definition of the infimum, $m\le f(x)$ and $n\le g(x)$ for all $x\in D$. Again, these two inequalities can be added together to show that $m+n\le f(x)+g(x)$ for all $x\in D$. This means that all pairs of $f$ and $g$ for all $x\in D$, their sum is greater than or equal to the sum of the individual function's infimum. Consequently, the greatest lower bound of $f(x)+g(x)$ must also be greater than or equal to $m+n$. By definition, the greatest lower bound $f(x)+g(x)\coloneq\underset{x\in D}{\inf}(f(x)+g(x))$. Substituting this expression into  $m+n\le f(x)+g(x)$, along with known values of $m$ and $n$, it can be seen that

\begin{align*}
	\underset{x\in D}{\inf}f(x) + \underset{x\in D}{\inf}g(x)\le \underset{x\in D}{\inf}(f(x)+g(x))
\end{align*}

\noindent which can be reversed to show

\begin{align*}
	\underset{x\in D}{\inf}(f(x)+g(x))\ge \underset{x\in D}{\inf}f(x) + \underset{x\in D}{\inf}g(x)
\end{align*}

Therefore if $D$ is a nonempty set, and $f : D\to\bR$ and $g : D\to \bR$ are bounded functions, then it holds that, for all $x\in D$,

\begin{align*}
	\underset{x\in D}{\sup}(f(x) + g(x))\le\underset{x\in D}{\sup}f(x) + \underset{x\in D}{\sup}g(x) \quad \text{and} \quad \underset{x\in D}{\inf}(f(x)+g(x))\ge \underset{x\in D}{\inf}f(x) + \underset{x\in D}{\inf}g(x)
\end{align*}

\end{proof}

\newpage

\noindent (b) Find an example where we obtain strict inequalities.

\begin{proof}

Suppose $D=[1,2]$, so $D$ is non-empty. Allow $f(x)$ and $g(x)$ to be function of $x\in D$. Let $f(x)=x$, and $g(x)=-3x+4$. We observe the infimum and supremum below by noting that $f(x)$ and $g(x)$ are constantly increasing and decreasing functions, respectively. This means that the infimum of $f$ is found at the minimum $x$ value in $D$, and the supremum at the maximum value. The opposite is true for $g$. Consequently,

\begin{align*}
	\underset{x\in D}{\inf}f(x)=\inf f(1) = 1 && \underset{x\in D}{\sup}f(x)=\sup f(2) = 2 \\
	\underset{x\in D}{\inf}g(x)=\inf g(2) = -2 && \underset{x\in D}{\sup}g(x)=\sup g(1) = 1
\end{align*}

To determine if strict inequalities hold for 

\begin{align*}
	\underset{x\in D}{\sup}(f(x) + g(x))\le\underset{x\in D}{\sup}f(x) + \underset{x\in D}{\sup}g(x) \quad \text{and} \quad \underset{x\in D}{\inf}(f(x)+g(x))\ge \underset{x\in D}{\inf}f(x) + \underset{x\in D}{\inf}g(x)
\end{align*}

\noindent we can simply plug in the respective values for the individual terms, though it requires more effort for the infimum/supremum of the sum of the functions. We can add $f(x)$ and $g(x)$ together and call their sum $h(x)$ as

\begin{align*}
	h(x)=f(x)+g(x)=x+(-3x+4)=-2x+4
\end{align*}

\noindent Like $g(x)$, $h(x)$ is constantly decreasing and finds its supremum at its minimum inputted $x$ value and the infimum at the maximum input. These are found to be

\begin{align*}
	\underset{x\in D}{\inf}h(x)=\inf h(2) = 0 && \underset{x\in D}{\sup}h(x)=\sup h(1) = 2
\end{align*}

Finally, we can substitute respective values for $\underset{x\in D}{\inf}f(x)$, $\underset{x\in D}{\sup}f(x)$, $\underset{x\in D}{\inf}g(x)$, $\underset{x\in D}{\sup}g(x)$, and the values found from $h(x)$ into $\underset{x\in D}{\inf}(f(x) + g(x))$ and $\underset{x\in D}{\sup}(f(x) + g(x))$ to show that

\begin{align*}
	2 \le& 2 + 1 && 0 \ge 1 + (-2) \\
	2\le& 3 && 0\ge -1 \\
	2 <& 3 && 0>-1
\end{align*}

As shown, there is no equality in the above expressions, so they are said to be strict. This is simply one of many examples, and others may be found without going as in depth.

\end{proof}
\end{document}
