\documentclass[12pt]{article}

% Custom Commands
\newcommand{\set}[1]{\left\{ {#1} \right\}}
\newcommand{\limit}[1]{\displaystyle \lim_{ {#1} }}
\newcommand{\limtoinf}[1][n]{\displaystyle\lim_{ {#1} \to \infty}}
\newcommand{\liminftoinf}[1][n]{\displaystyle\liminf_{ {#1} \to \infty}}
\newcommand{\limsuptoinf}[1][n]{\displaystyle\limsup_{ {#1} \to \infty}}
\newcommand{\abs}[1]{\left| {#1} \right|}
\newcommand{\ceil}[1]{\lceil {#1} \rceil}
\newcommand{\floor}[1]{\lfloor {#1} \rfloor}
\newcommand{\seq}[2][n]{\left\{ {#2} \right\}_{#1=1}^\infty}
\newcommand{\paren}[1]{\left( {#1} \right)}
\newcommand{\series}[2]{\displaystyle \sum_{ {#1} }^{ {#2} }}
\newcommand{\bR}{\mathbb{R}}
\newcommand{\bZ}{\mathbb{Z}}
\newcommand{\bN}{\mathbb{N}}
\newcommand{\bQ}{\mathbb{Q}}

% Math and symbol packages
\usepackage{amsmath}
\usepackage{amsthm}
\usepackage{amssymb}
\usepackage{mathtools}
\usepackage{derivative}

% Formatting
\usepackage{inputenc}
\usepackage[left=2.54cm,right=2.54cm,top=2.54cm,bottom=2.54cm]{geometry}
\usepackage{fancyhdr}
\usepackage{lipsum}

% Spacing formats
\AtBeginDocument{
	\setlength{\abovedisplayskip}{-2pt}
	\setlength{\belowdisplayskip}{5pt}
}

% Actual content
\begin{document}
\pagestyle{fancy}
\setlength{\headheight}{14.49998pt}
\fancyhead[L]{Trevor Swan}
\fancyhead[C]{\textbf{MATH321 - E2 Definitions}}
\fancyhead[R]{10/28/24}
\fancyfoot[C]{\thepage}

\noindent \textbf{Facts about Sequences}

\begin{enumerate}
	\item Let $\seq{a_n}$ and $\seq{b_n}$ be such that $a_n\le x_n\le b_n$ for all $n\in\bN$. If $\seq{a_n}$ and $\seq{b_n}$ both converge to $x\in\bR$, then $\seq{x_n}$ must also converge to $x$.
	\item Let $\seq{x_n}, \seq{y_n}$ be convergent sequences. Suppose that $x_n\le y_n$ for every $n\in\bN$. Then $\limtoinf x_n \le \limtoinf y_n$.
	\item If $\seq{x_n}$ is a convergent sequence such that $x_n\ge0$ for all $n\in\bN$, then $\limtoinf x_n\ge0$.
	\item Let $a,b\in\bR$ and $\seq{x_n}$ be a convergent sequence such that $a\le x_n\le b$ for all $n\in\bN$. Then $a\le \limtoinf x_n\le b$.
	\item Addition, Subtraction, Multiplication, and Division can all be performed safely on limit expression provided they are both convergent and there is no division by 0.
	\item Let $\seq{x_n}$ be a convergent sequence such that $x_n\ge 0$ for all $n\in\bN$. then $\limtoinf \sqrt{x_n}=\sqrt{\limtoinf x_n}$.
	\item If $\seq{x_n}$ is a convergent sequence, then $\seq{\abs{x_n}}$ is a convergent sequence and $\limtoinf\abs{x_n}=\abs{\limtoinf x_n}$.
\end{enumerate}

\noindent \textbf{Convergence Tests}

\begin{enumerate}
	\item Let $\seq{x_n}$ be a sequence, and suppose there is an $x\in\bR$ and a sequence $\seq{a_n}$ such that $\limtoinf a_n =0$ and $\abs{x_n-x}\le a_n$ for all $n\in\bN$. Then the sequence $x_n$ converges to $x$.
	\item Let $c>0$. (1) If $c<1$, then $\seq{c^n}$ converges to zero. (2) If $c>1$, then $\set{c^n : n\in\bN}$ is an unbounded set of $\bR$ and hence cannot converge.
	\item Let $\seq{x_n}$ be a sequence such that $x_n\neq 0$ for all $n\in\bN$ and the limit $L\coloneq\limtoinf\abs{\frac{x_{n+1}}{x_n}}$ exists. Then, (1) If $0\le L<1$, then $\seq{x_n}$ converges to 0. (2) If $L>1$, then $\set{x_n : n\in\bN}$ is unbounded, and so $\seq{x_n}$ is divergent. (3) If $L=1$, the test is inconclusive.
\end{enumerate}

\noindent \textbf{Limit Superior \& Inferior}

\begin{enumerate}
	\item Let $\seq{x_n}$ be a bounded sequence. For each $n\in\bN$, define $a_n=\sup\set{x_k : k\ge n}$ and $b_n=\inf\set{x_k : k \ge n}$. Consider the sequences, $\seq{a_n}$ and $\seq{b_n}$. Define $\limsuptoinf x_n = \limtoinf a_n$ and $\liminftoinf x_n = \limtoinf b_n$, provided both limits exist.
	\item Let $\seq{x_n}$ be a bounded sequence, and $\seq{a_n}$, $\seq{b_n}$ be as in the above definition. (1) The sequence $\seq{a_n}$ is bounded and monotone decreasing, while $\seq{b_n}$ is also bounded but is monotone increasing. In particular $\limsuptoinf x_n$ and $\liminftoinf x_n$ both exist. (2)$\limsuptoinf x_n=\inf\set{a_n : n\in\bN}=\inf\set{\sup\set{a_k : k\ge n} : n\in\bN}$ and $\liminftoinf x_n = \sup{b_n : n\in\bN}=\sup\set{\inf\set{b_k : k\ge n} : n\in\bN}$. (3) $\limsuptoinf x_n\ge\liminftoinf x_n$.
	\item Given a bounded sequence $\seq{x_n}$, there exists subsequences $\seq[k]{x_{n_k}}$ and $\seq[k]{x_{m_k}}$ of $\seq{x_n}$ such that $\limit{k\to\infty} x_{n_k}=\limsuptoinf x_n$ and $\limit{k\to\infty} x_{m_k}=\liminftoinf x_n$.
	\item A bounded sequence $\seq{x_n}$ converges if and only if $\limsuptoinf x_n=\liminftoinf x_n$. In fact, $\limsuptoinf x_n=\limtoinf x_n=\liminftoinf x_n$.
	\item Suppose you have a bounded sequence $\seq{x_n}$ and $\seq[k]{x_{n_k}}$ is a subsequence. Then $\liminftoinf x_n\le\liminf_{k\to\infty} x_{n_k}\le\limsup_{k\to\infty} x_{n_k}\le\limsuptoinf x_n$.
	\item A bounded sequence $\seq{x_n}$ converges to $x\in\bR$ if and only if every convergent subsequence converges to $x$.
\end{enumerate}

\noindent \textbf{Bolzano-Weierstrass Theorem}

\begin{enumerate}
	\item Every bounded sequence of real number must have a convergent subsequence.
	\item We say that a sequence $\seq{x_n}$ diverges to infinity if, for every $k\in\bR$, there is some $M$ such that $x_n>k$ whenever $n\ge M$. In this case, we write $\limtoinf x_n\coloneq\infty$.
	\item We say that a sequence $\seq{x_n}$ diverges to negative infinity if, for every $k\in\bR$, there is some $M$ such that $x_n<k$ whenever $n\ge M$. In this case, we write $\limtoinf x_n\coloneq-\infty$.
	\item Let $\seq{x_n}$ be an unbounded sequence of real numbers. Define the sequence of extended real number $\seq{a_n}$ and $\seq{b_n}$ by $a_n=\sup\set{x_k : k\ge n}$, $b_n=\inf\set{x_k : k\ge n}$. If each $a_n$ and $b_n$ is a real number, then $\limsuptoinf x_n=\limtoinf a_n$ and $\liminftoinf x_n = \limtoinf b_n$.
\end{enumerate}

\noindent \textbf{Cauchy Sequences}

\begin{enumerate}
	\item Let $\seq{x_n}$ be a sequence of real numbers. Then $\seq{x_n}$ is said to be a cauchy sequence if, for every $\epsilon>0$, there is some $M\in\bN$ such that $\abs{x_n-x_M}<M\epsilon$ whenever $n,m\ge M$.
	\item Cauchy sequences are bounded.
	\item A sequence of real numbers is convergent if and only if it is cauchy.
\end{enumerate}

\noindent \textbf{Series}

\begin{enumerate}
	\item Given a sequence $\seq{x_n}$, we write the formal object $\series{n=0}{\infty} x_n$ and call it a series. A series $\series{n=0}{\infty} x_n$ converges if the sequence $\seq[k]{S_k}$ given by $S^k\coloneq\series{n=1}{k} x_n = x_1+x_2+x_3+\dots+x_k$ converges. The numbers $S_k$ are called partial sums. If $\series{n=0}{\infty} x_n$ should converge, we write $\series{n=0}{\infty} x_n=\limit{k\to\infty} S^k=\limit{k\to\infty}\series{n=0}{k} x_n$. If the sequence $\seq[k]{S^k}$ diverges, then we say that $\series{n=0}{\infty} x_n$ diverges.
	\item Let $r\in\bR$. The geometric series $\series{n=0}{\infty} r^n$ converges if and only if $-1<r<1$. In particular $\series{n=0}{\infty} r^n=\frac{1}{1-r}$ given that $-1<r<1$ and the series is convergent.
	\item A series $\series{n=0}{\infty} x_n$ converges if and only if its tails converge (i.e for $M\in\bN$. the series $\series{n=M}{\infty} x_n$ converges).
	\item A series $\series{n=0}{\infty} x_n$ is said to be cauchy if the sequence of partial sums $\seq[k]{S^k}$ is a cauchy sequence. For every $\epsilon > 0$, there exists an $M\in\bN$ such that $\abs{\series{k=1}{m} x_n-\series{k=1}{k} x_n}<\epsilon$ whenever $m, k\ge M$. Without loss of generality, we may suppose that $K>m$. Then we may write $\abs{\series{n=m+1}{k} x_n}<\epsilon$ for all $k,m\ge M$.
	\item A series is cauchy if and only if for every $\epsilon>0$ there is an $M\in\bN$ such that $\abs{\series{n=m+1}{k} x_n}<\epsilon$ whenever $k,m \ge M$.
	\item Let $\series{n=0}{\infty} x_n$ be a convergent series. Then the sequence of terms $\seq{x_n}$ converges and in fact $\limtoinf x_n=0$. The converse is $\underline{\text{not}}$ true.
	\item The summation is a linear operator.
	\item If $x_n\ge 0$ for all $n\in\bN$, then $\series{n=0}{\infty} x_n$ converges if and only if the sequence of partial sums in bounded above.
	\item A series $\series{n=0}{\infty} x_n$ converges absolutely if $\series{n=0}{\infty} \abs{x_n}$ converges. If a series converges, but doesn't converge absolutely, we say that it converges conditionally.
	\item If $\series{n=0}{\infty} x_n$ converges absolutely, then it also converges conditionally. The converse is $\underline{\text{not}}$ true.
\end{enumerate}

\noindent \textbf{Series Tests}

\begin{enumerate}
	\item \textit{Comparison Test:} Let $\series{n=0}{\infty} x_n$ and $\series{n=0}{\infty} y_n$ be a pair of series such that $0\le x_n\le y_n$ for all $n\in\bN$. (1) If $\series{n=0}{\infty} y_n$ converges, then $\series{n=0}{\infty} x_n$ converges. (2) If $\series{n=0}{\infty} x_n$ diverges, then $\series{n=0}{\infty} y_n$ diverges.
	\item \textit{P-series Test:} For $p\in\bR$, the series given by $\series{n=0}{\infty} \frac{1}{n^p}$ converges if and only if $p>1$.
	\item \text{Ratio Test:} Let $\series{n=0}{\infty} x_n$ be a series such that $x_n\neq 0$ for every $n\in\bN$, and such that $L\coloneq\limtoinf\frac{\abs{x_{n+1}}}{\abs{x_n}}$. If (1) If $L<1$, then $\series{n=0}{\infty} x_n$ converges absolutely. (2) If $L>1$, then $\series{n=0}{\infty} x_n$ diverges. (3) If $L=1$, then the test is inconclusive.
\end{enumerate}

\noindent \textbf{Limits of Functions}

\begin{enumerate}
	\item Let $S\subset\bR$. A number $x\in\bR$ is called a cluster, or limit, point of $S$ if, for every $\epsilon>0$, the set $\paren{x-\epsilon, x+\epsilon}\cap [ S\setminus \set{x}]$ is nonempty. In other words, $x$ is a cluster point of $S$ if for every $\epsilon>0$, there is some $y\in S$ with $y\neq x$, such that $\abs{y-x}<\epsilon$, requiring that $y\in\paren{x-\epsilon, x+\epsilon}\cap [S\setminus\set{x}]$.
	\item Let $S$ be a subset of $\bR$ and $c\in\bR$, then $c$ is a cluster point of $S$ if and only if there is a sequence $\seq{x_n}$ with $x_n\in S\setminus \set{c}$ such that $\limtoinf x_n=c$.
	\item Let $f:S\to\bR$, with $S\subset\bR$ be non empty and suppose that $c\in\bR$ is a cluster point of $S$. Suppose $L\in\bR$ is such that, for every $\epsilon>0$, there exists some $\delta>0$ for which $\abs{f(x)-L}<\epsilon$ holds whenever $x\in S\setminus\set{c}$ satisfies $\abs{x-c}<\delta$. We then say that $f(x)$ converges to $L$ as $x\to c$, and we write $f(x)\to L$ as $x\to c$. We can this $L$ a limit of $f(x)$ as $x$ goes to $c$, and if $L$ is unique we write $\limit{x\to c}f(x)=L$. If no such $L$ exists, we say that $f$ diverges at $c$.
	\item Let $c$ be a cluster point of $S\in\bR$ and let $f:S\to\bR$ be a function such that $f(x)$ converges as $x$ goes to $c$. Then the limit of $f(x)$ as $x\to c$ is unique.
	\item Let $S\subset\bR$, $c$ be a cluster points of $S$, $f:S\to\bR$ be a function, and $L\in\bR$. Then $f(x)\to L$ as $x\to c$ if and only if for every sequence $\seq{x_n}$ such that $x_n\in S\setminus\set{c}$ for all $n$, and such that $\limtoinf x_n =c$, we have that the sequence $\seq{f(x_n)}$ converges to $L$.
	\item $S\subset\bR$, $c$ be a cluster point of $S$, $f:S\to\bR$ and $L\in\bR$. Then $f(x)\to L$ as $x\to c$ if for every $\epsilon>0$, there exists some $\delta>0$ such that $\abs{f(x)-L}<\epsilon$ whenever $x\in S\setminus\set{c}$ such that $\abs{x-c}<\delta$.
	\item Let $S\subset\bR$ and $c$ be a cluster point of $S$. Suppose $f:S\to\bR$ is a function such that the limit of $f(x)$ as $x\to c$ exists. Suppose there are two real numbers $a,b\in\bR$ with $a\le f(x)\le b$ for all $x\in S\setminus\set{c}$. Then $a\le\limit{x\to c} f(x)\le b$.
	\item Let $S\subset\bR$ and $c$ be a cluster point of $S$. Suppose $f:S\to\bR$ and $g:S\to\bR$ are functions such that the limits as $x\to c$ both exist. If $f(x)\le g(x)$ holds for every $x\in S\setminus\set{c}$ then $\limit{x\to c} f(x)\le \limit{x\to c} g(x)$.
	\item Limits are preserved across Addition, Subtraction, Multiplication, Division, and Absolute value provided there is no division by 0 and that both limits exist.
	\item Let $f:S\to\bR$ be a function and let $A\subset S\subset\bR$. Define the function $f|_A=f(x)$ for $x\in A$. We call $f|_A$ the restriction of $f$ to $A$.
	\item Let $S\subset\bR$, $c\in\bR$, and $f:S\to\bR$ be a function. Suppose $A\subset S$ is such that there is some $\alpha>0$ satisfying $[A\setminus\set{c}]\cap\paren{c-\alpha, c+\alpha}=[S\setminus\set{c}]\cap\paren{c-\alpha, c+\alpha}$. Then (1) The point $c$ is a cluster point of $A$ if and only if $c$ is a cluster point of $S$. (2) Supposing $c$ is a cluster point of $S$, then $f(x)\to L$ as $x\to c$ if and only if $f|_A\to L$ as $x\to c$.
\end{enumerate}

\noindent \textbf{Continuous Functions}

\begin{enumerate}
	\item Suppose $S\subset\bR$ and $c\in S$. We say $f:S\to\bR$ is continuous at $c$ is for every $\epsilon>0$ there is a $\delta>0$ such that whenever $x\in S$ and $\abs{x-c}<\delta$, we have $\abs{f(x)-f(c)}<\epsilon$. When $f$ is continuous at all $c\in S$, then we say that $f$ is a continuous function. If $f$ is continuous for all $c\in A$, we say $f$ is continuous on $A\subset S$. This implies that $f|_A$ is continuous, but the converse does not hold.
	\item If for $f:S\to\bR$ and $A\subset S$, $f$ is continuous, then $f|_A$ is also continuous. The converse is false.
	\item Consider a function $f:S\to\bR$ defined on a set $S\subset\bR$ and let $c\in S$. Then (1) If $c$ is not a cluster point of $S$, then $f$ is continuous at $c$. (2) If $c$ is a cluster point of $S$, then $f$ is continuous at $c$ if and only if the limit of $f(x)$ as $x\to c$ exists and $\limit{x\to c}f(x)=f(c)$. (3) The function $f$ is continuous at $c$ if and only if for every sequence $\seq{x_n}$ where $x_n\in S$ and $\limtoinf x_n=c$, the sequence $\seq{f(x_n)}$ converges to $f(c)$.
	\item The third statement above allows us to quickly apply what we know about limits of sequences to continuous functions and even prove that certain functions are continuous.
	\item The Addition, Subtraction, Multiplication, and Division of functions continuous at some $c\in S$ results in a function continuous at $c$, given that there is no division by 0.
	\item All polynomials are continuous.
	\item Let $A,B\subset\bR$ and $f:B\to\bR$ and $g:A\to B$ be functions. If $g$ is continuous at $c\in A$ and $f$ is continuous at $g(c)$, then $f \circ g=f(g(x)):A\to B$ is continuous at $c$.
	\item Discontinuity at a point $c$ is true when $f$ is not continuous at $c$.
	\item Let $f:S\to\bR$ be a function and $c\in S$. Suppose that there exits a sequence $\seq{x_n}$, $x_n\in S$ for all $n$, and $\limtoinf x_n=c$ such that $\seq{f(x_n)}$ does not converge to $f(c)$. Then $f$ is discontinuous at $c$.
	\item \[f(x)\coloneq
	\begin{cases}
		1 & \text{if }x\text{ is rational} \\
		0 & \text{if }x\text{ is irrational}
	\end{cases}
	\] This function is discontinuous at all $c\in\bR$.
	\item \[f(x)\coloneq
	\begin{cases}
		\frac{1}{k} & \text{if }x\text{ is rational and in lowest terms} \\
		0 & \text{if }x\text{ is irrational}
	\end{cases}
	\] This function is irrational $c$ but discontinuous at all rational $c$.
	\item A point is called a removable discontinuity if we could change the definition of its function by insisting that the point takes on a different value and obtain a continuous function.
\end{enumerate}

\noindent \textbf{Extreme Value Theorem}

\begin{enumerate}
	\item $f[a,b]\to\bR$ is bounded is there exists a $B\in\bR$ such that $\abs{f(x)}\le B$ for every $x\in[a,b]$.
	\item A continuous function on a compact interval $f[a,b]\to\bR$ is necessarily bounded.
	\item (1) $f:S\to\bR$ achieves an absolute minimum at $c\in S$ if $f(c)\le f(x)$ for every $x\in S$. (2) $f:S\to\bR$ achieves an absolute maximum at $c\in S$ if $f(c)\ge f(x)$ for every $x\in S$.
	\item \textit{Extreme Value Theorem:} A continuous function $f[a,b]\to\bR$ achieves both an absolute minimum and an absolute maximum on $[a,b]$.
	\item A compact interval $[a,b]$ is essential to the validity of the EVT. Continuity of $f$ is also essential.
\end{enumerate}

\end{document}
