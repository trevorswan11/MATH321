\documentclass[12pt]{article}

% Custom Commands
\newcommand{\set}[1]{\left\{ {#1} \right\}}
\newcommand{\limtoinf}[1]{\lim_{ {#1} \to\infty}}
\newcommand{\abs}[1]{\left| {#1} \right|}
\newcommand{\ceil}[1]{\lceil {#1} \rceil}
\newcommand{\floor}[1]{\lfloor {#1} \rfloor}
\newcommand{\seq}[1]{\set{ {#1} }_{n=1}^\infty}
\newcommand{\paren}[1]{\left( {#1} \right)}
\newcommand{\bR}{\mathbb{R}}
\newcommand{\bZ}{\mathbb{Z}}
\newcommand{\bN}{\mathbb{N}}
\newcommand{\bQ}{\mathbb{Q}}

% Math and symbol packages
\usepackage{amsmath}
\usepackage{amsthm}
\usepackage{amssymb}
\usepackage{mathtools}
\usepackage{derivative}

% Formatting
\usepackage{inputenc}
\usepackage[left=2.54cm,right=2.54cm,top=2.54cm,bottom=2.54cm]{geometry}
\usepackage{fancyhdr}
\usepackage{lipsum}

% Actual content
\begin{document}
\pagestyle{fancy}
\setlength{\headheight}{14.49998pt}
\fancyhead[L]{Trevor Swan}
\fancyhead[C]{\textbf{MATH321 - HW3}}
\fancyhead[R]{09/18/24}
\fancyfoot[C]{\thepage}

% Start Problem 1
\noindent In the following exercise, feel free to use what you know from calculus to find the limit, if it exists. But you must \textit{prove} that you found the correct limit, or that the sequence is divergent. \\

\noindent \textbf{(2.1.5)} Is the sequence $\set{\frac{n}{n+1}}_{n=1}^\infty$ convergent? If so, what is the limit?

\noindent \underline{Limit Calculation with L'Hopital Rule}
\begin{align*}
	\limtoinf{n} \frac{n}{n+1} \to & \frac{\infty}{\infty} \\
	\overset{LH}{=}& \limtoinf{n} \frac{\frac{d}{dn}(n)}{\frac{d}{dn}(n+1)} \\
	=& \limtoinf{n} \frac{1}{1} \\
	=& 1
\end{align*}

\noindent \underline{Scratch Work}
\begin{equation*}
	\abs{\frac{n}{n+1}-1}=\abs{\frac{n}{n+1}-\frac{n+1}{n+1}}=\abs{\frac{n-(n+1)}{n+1}}=\abs{\frac{-1}{n+1}}
\end{equation*}

\begin{proof}
	$\set{\frac{n}{n+1}}_{n=1}^\infty$ is said to converge to some $x\in\bR$ if for every $\epsilon>0$, there exists an $M\in\bN$ such that $\abs{x_n-x}<\epsilon$ whenever $n\ge M$.
	
\indent Let $\epsilon > 0$ be given and choose $x=1$, so $\abs{\frac{n}{n+1}-1}<\epsilon$. If $x_n$ is convergent, this is true for all $n\ge M$ where $M\in\bN$. $n$ is strictly positive as $n\in\bN$, so by using the Scratch Work above it is said that

\begin{equation*}
	\abs{x_n-x}=\abs{\frac{-1}{n+1}}=\frac{1}{n+1}
\end{equation*}

\noindent By the definition of a convergent sequence, it must be shown that $\frac{1}{n+1}<\epsilon$ when $\epsilon >0$ for some $n\ge M$, $M\in\bN$. The desired inequality can be written as $\frac{1}{n+1}<\epsilon\equiv n+1>\frac{1}{\epsilon}\equiv n>\frac{1}{\epsilon}-1$. Choose $M=\ceil{\frac{1}{\epsilon}-1}$, then for all $n\ge M$, it is true that $\frac{1}{n+1}<\epsilon$ and that $\limtoinf{n} \frac{n}{n+1}=1$. Thus the sequence $\set{\frac{n}{n+1}}_{n=1}^\infty$ is convergent and converges to $1$. 
\end{proof}

\newpage

% Start Problem 2
\noindent \textbf{(2.1.9)} Show that the sequence $\set{\frac{1}{\sqrt[3]{n}}}_{n=1}^\infty$ is monotone and unbound. Then use \underline{Theorem 2.1.10}, also known as the Monotone Convergence Theorem (MCT), to find the limit.

\noindent \underline{Show $x^\frac{1}{3}$ is an Increasing Function} \\
\noindent Suppose $f(x)=\sqrt[3]{x}=x^\frac{1}{3}$. Then, $f'(x) = \frac{d}{dx}\left(x^{1/3}\right) = \frac{1}{3}x^{-2/3} = \frac{1}{3\sqrt[3]{x^2}}$

\begin{enumerate}
	\item For  $x > 0$, $\sqrt[3]{x^2}$ is positive, so $f'(x) = \frac{1}{3\sqrt[3]{x^2}} > 0.$
	\item 
For $x < 0$, $\sqrt[3]{x^2}$ is also positive, and hence $f'(x) = \frac{1}{3\sqrt[3]{x^2}} > 0.$
\end{enumerate}

\noindent In both cases, the derivative $f'(x)$ is positive, indicating that $f(x)$ is an increasing function.
\begin{proof}
	$\seq{\frac{1}{\sqrt[3]{n}}}$ is given. Consider, for some arbitrary $n$, $x_n$ and $x_{n+1}$. These two values of the sequence can be compared, and if $x_n>x_{n+1}$, then $\seq{x_n}$ is decreasing.
	
\begin{align*}
	x_n>x_{n+1}\equiv&\frac{1}{\sqrt[3]{n}}>\frac{1}{\sqrt[3]{n+1}} && \text{Substitute Given} \\
	\equiv& \sqrt[3]{n+1} > \sqrt[3]{n} && \text{Cross Multiply} \\
\end{align*}

\noindent It is shown above that $x^\frac{1}{3}$ is an increasing function, so the statement $\sqrt[3]{n+1} > \sqrt[3]{n}$ is true. Consequently, this also proves that $x_n>x_{n+1}$. Thus, $\seq{\frac{1}{\sqrt[3]{n}}}$ is monotone decreasing as $n$ is arbitrary. Since $\seq{\frac{1}{\sqrt[3]{n}}}$ is monotone decreasing, it is said to be monotone.

\indent Take $\seq{\frac{1}{\sqrt[3]{n}}}$ as given, and examine the function which defines the sequence $\frac{1}{\sqrt[3]{n}}=n^{-\frac{1}{3}}$. If the sequence is unbounded, then $n^{-\frac{1}{3}}>M$ for some arbitrarily large $M$. Solve for $n$ as

\begin{align*}
	n^{-\frac{1}{3}}>&M && \text{Given} \\
	n^{\frac{1}{3}}<&\frac{1}{M} && \text{Take the reciprocal of both sides} \\
	n <& \paren{\frac{1}{M}}^3 && \text{Cube both sides}
\end{align*}

\noindent For any $M>0$, it is possible to find an $n\in\bR$ such that $n<\paren{\frac{1}{M}}^3$, and thus the original sequence inequality holds that $\frac{1}{\sqrt[3]{n}}>M$ for any $n$ and arbitrarily large $M$. Hence the sequence $\seq{\frac{1}{\sqrt[3]{n}}}$ is unbounded.

\indent The MCT states that if a sequence is monotone decreasing and bounded, then $\limtoinf{n}x_n = \inf\set{x_n:n\in\bN}$. As $n\to\infty, \frac{1}{\sqrt[3]{n}}\to0$ as $\seq{\frac{1}{\sqrt[3]{n}}}$ is monotone decreasing and $n\in\bN$. This means that 0 is the sequence's greatest lower bound (infimum) as it approaches but never reaches 0 as $n$ gets arbitrarily large. Therefore $\seq{\frac{1}{\sqrt[3]{n}}}$ is bounded below and $\limtoinf{n}\frac{1}{\sqrt[3]{n}} = \inf\set{x_n:n\in\bN}=0$.
\end{proof}

\newpage

% Start Problem 3
\noindent \textbf{(2.1.12)} Prove \underline{Proposition 2.1.13}:

\indent Let $S\subset\bR$ be a nonempty bounded set. Then there exist monotone sequences $\set{x_n}_{n=1}^\infty$ and $\set{y_n}_{n=1}^\infty $ such that $x_n, y_n\in S$ and

\begin{align*}
	\sup S = \lim_{n\to\infty} x_n \quad \text{and} \quad \inf S = \lim_{n\to\infty}.
\end{align*}

\begin{proof}
	It is given that $S$ is nonempty and bounded, meaning that both $\sup S$ and $\inf S$ exist.

\indent Suppose $x_n$ is a sequence whose elements are a subset of $S$. 
\end{proof}

\newpage

% Start Problem 4
\noindent \textbf{(2.1.15)} Let $\set{x_n}_{n=1}^\infty$ be a sequence defined by

\begin{align*}
	x_n\coloneq \begin{cases}
             n & \text{if } n \text{ is odd}, \\
             \frac{1}{n} & \text{if } n \text{ is even}.
        		\end{cases}
\end{align*}

\noindent (a) Is the sequence bounded? (prove or disprove)

\begin{proof} The sequence given is $\seq{x_n}$ and it has two branches which leads to two different cases as follows:
	\begin{enumerate}
		\item Suppose $n$ is odd. Corresponding terms in the given sequence $\seq{x_n}$ are $1,3,7,\dots,n$.
		\item Suppose $n$ is even. Corresponding terms in the given sequence $\seq{x_n}$ are $\frac{1}{2}, \frac{1}{4}, \frac{1}{6},\dots, \frac{1}{n}$.
	\end{enumerate}
	In case 1, the corresponding terms are the set of all positive odd numbers. Assume to the contrary that there exists a greatest odd integer, say $M=2m+1$, for some $m\in\bZ$. Take the integer $M'$, who is represented by $M'=M+2=2m+3$. $M'$ is clearly greater than $M$, so there is no greatest odd number. Since this subsequence of $x_n$ is unbounded, then it implies that the entire sequence is unbounded. This proves that when $n$ is odd, then the given sequence is unbounded and approaches infinity.
\end{proof}

\noindent (b) Is there a convergent subsequence? If so, find it.

\begin{proof}
	It has been shown in part (a) that when $n$ is odd, this subsequence is unbounded and diverges to infinity. Instead, take the subsequence when $n$ is even, so $\set{x_n}_{n=2k}^\infty$ for $k\in\bN$. As $n$ increases it is observed that the terms get arbitrarily small and approach 0, which we will assume for now is the limit. Let $\epsilon>0$ be given. By the archimedean property, there must exists some $M\in\bN$ such that $0<\frac{1}{M}<\epsilon$. Consequently, for every $n\ge M$, we have that $\abs{x_n-0}=\abs{\frac{1}{n}}\le \frac{1}{M}<\epsilon$ as required. Therefore the subsequence is convergent to 0 and the sequence given has a convergent subsequence as shown.
\end{proof}

\newpage

% Start Problem 5
\noindent \textbf{(2.1.23)} Suppose that $\set{x_n}_{n=1}^\infty$ is a monotone increasing sequence that has a convergent subsequence. Show that $\set{x_n}_{n=1}^\infty$ is convergent. \textit{Note that \underline{Proposition 2.1.17} is an "if and only if" for monotone sequences.}

\begin{proof}
	\lipsum[1]
\end{proof}

\end{document}
