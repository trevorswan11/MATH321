\documentclass[12pt]{article}

% Custom Commands
\newcommand{\set}[1]{\left\{ {#1} \right\}}
\newcommand{\limtoinf}[1]{\lim_{ {#1} \to\infty}}
\newcommand{\abs}[1]{\left| {#1} \right|}
\newcommand{\ceil}[1]{\lceil {#1} \rceil}
\newcommand{\floor}[1]{\lfloor {#1} \rfloor}
\newcommand{\seq}[2][n]{\left\{ {#2} \right\}_{#1=1}^\infty}
\newcommand{\paren}[1]{\left( {#1} \right)}
\newcommand{\bR}{\mathbb{R}}
\newcommand{\bZ}{\mathbb{Z}}
\newcommand{\bN}{\mathbb{N}}
\newcommand{\bQ}{\mathbb{Q}}

% Math and symbol packages
\usepackage{amsmath}
\usepackage{amsthm}
\usepackage{amssymb}
\usepackage{mathtools}
\usepackage{derivative}

% Formatting
\usepackage{inputenc}
\usepackage[left=2.54cm,right=2.54cm,top=2.54cm,bottom=2.54cm]{geometry}
\usepackage{fancyhdr}
\usepackage{lipsum}

% Actual content
\begin{document}
\pagestyle{fancy}
\setlength{\headheight}{14.49998pt}
\fancyhead[L]{Trevor Swan}
\fancyhead[C]{\textbf{MATH321 - HW4}}
\fancyhead[R]{09/27/24}
\fancyfoot[C]{\thepage}

% Begin Problem 1
\noindent \textbf{(2.2.3)} Prove that if $\seq{x_n}$ is a convergent sequence, $k\in\bN$, then

\begin{equation*}
	\limtoinf{n} x_n^k = \paren{\limtoinf{n} x_n}^k.
\end{equation*}

\begin{proof}
	\lipsum[1]
\end{proof}

\newpage

% Begin Problem 2
\noindent \textbf{(2.2.5)} Let $x_n\coloneq\frac{n-\cos(n)}{n}$. Use the squeeze lemma to show that $\seq{x_n}$ converges and find the limit.

\begin{proof}
	\lipsum[1]
\end{proof}

\newpage

% Begin Problem 3
\noindent \textbf{(2.2.9)} Suppose $\seq{x_n}$ is a sequence, $x\in\bR$, and $x_n\neq x$ for all $n\in\bN$. Suppose the limit

\begin{equation*}
	L\coloneq\limtoinf{n}\frac{\abs{x_{n+1}-x}}{\abs{x_n-x}}
\end{equation*}

\noindent exists and $L<1$. Show that $\seq{x_n}$ converges to x.

\begin{proof}
	\lipsum[1]
\end{proof}

\newpage

% Begin Problem 4
\noindent \textbf{(2.2.12)} Let $\seq{a_n}$ and $\seq{b_n}$ be sequences.

\noindent a) Suppose $\seq{a_n}$ is bounded and $\seq{b_n}$ converges to 0. Show that $\seq{a_nb_n}$ converges to 0.

\begin{proof}
	\lipsum[1]
\end{proof}

\noindent b) Find an example where $\seq{a_n}$ is unbounded, $\seq{b_n}$ converges to 0, and $\seq{a_nb_n}$ is not convergent.

\begin{proof}
	\lipsum[1]
\end{proof}

\noindent c) Find an example where $\seq{a_n}$ is bounded, $\seq{b_n}$ converges to some $x\neq0$, and $\seq{a_nb_n}$ is not convergent.

\begin{proof}
	\lipsum[1]
\end{proof}

\newpage

% Begin Problem 5
\noindent \textbf{(2.2.15)} Prove $\limtoinf{n}\paren{n^2+1}^\frac{1}{n}=1$

\begin{proof}
	\lipsum[1]
\end{proof}

\newpage

\end{document}
