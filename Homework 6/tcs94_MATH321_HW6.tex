\documentclass[12pt]{article}

% Custom Commands
\newcommand{\set}[1]{\left\{ {#1} \right\}}
\newcommand{\limit}[1]{\displaystyle \lim_{ {#1} }}
\newcommand{\limtoinf}[1][n]{\displaystyle\lim_{ {#1} \to \infty}}
\newcommand{\liminftoinf}[1][n]{\displaystyle\liminf_{ {#1} \to \infty}}
\newcommand{\limsuptoinf}[1][n]{\displaystyle\limsup_{ {#1} \to \infty}}
\newcommand{\abs}[1]{\left| {#1} \right|}
\newcommand{\ceil}[1]{\lceil {#1} \rceil}
\newcommand{\floor}[1]{\lfloor {#1} \rfloor}
\newcommand{\seq}[2][n]{\left\{ {#2} \right\}_{#1=1}^\infty}
\newcommand{\paren}[1]{\left( {#1} \right)}
\newcommand{\series}[2]{\displaystyle \sum_{ {#1} }^{ {#2} }}
\newcommand{\bR}{\mathbb{R}}
\newcommand{\bZ}{\mathbb{Z}}
\newcommand{\bN}{\mathbb{N}}
\newcommand{\bQ}{\mathbb{Q}}

% Math and symbol packages
\usepackage{amsmath}
\usepackage{amsthm}
\usepackage{amssymb}
\usepackage{mathtools}
\usepackage{derivative}

% Formatting
\usepackage{inputenc}
\usepackage[left=2.54cm,right=2.54cm,top=2.54cm,bottom=2.54cm]{geometry}
\usepackage{fancyhdr}
\usepackage{lipsum}

% Spacing formats
\AtBeginDocument{
	\setlength{\abovedisplayskip}{-2pt}
	\setlength{\belowdisplayskip}{5pt}
}

% Actual content
\begin{document}
\pagestyle{fancy}
\setlength{\headheight}{14.49998pt}
\fancyhead[L]{Trevor Swan}
\fancyhead[C]{\textbf{MATH321 - HW6}}
\fancyhead[R]{10/11/24}
\fancyfoot[C]{\thepage}

% Begin Problem 1
\noindent \textbf{(2.5.2)} Prove \underline{Proposition 2.5.5}, that is, for $-1<r<1$, prove

\begin{align*}
	\series{n=0}{\infty}r^n=\frac{1}{1-r}
\end{align*}

\noindent given that the geometric series $\series{n=0}{\infty}r^n$ converges.

\begin{proof}
	Let $S_k$ be the partial sum of the series up to the $k$-th term of the series, that is $S_k=\sum{n=0}{\infty}=1+r+r^2+\dots+r^k$. We must first show that $S_k=\frac{1-r^{k+1}}{1-r}$ for a finite $k$ and $r\neq1$. For the base case, take $k=0$ so $S_0=1$ and the formula gives $S_0=\frac{1-r^1}{1-r}=1$, so this holds for $k=0$. Assume this holds for some $k$, so $S_k=1+r+r^2+\dots+r^k=\frac{1-r^{k+1}}{1-r}$. We must show that this holds for $k+1$, or that $S_{k+1}=1+r+r^2+\dots+r^{k+1}=\frac{1-r^{k+2}}{1-r}$. Notice that $S_{k+1}=S_k+r^{k+1}$, so we have $S_{k+1}=\frac{1-r^{k+1}}{1-r}+r^{k+1}$. We can then simply this expression as
	
\begin{align*}
	S_{k+1}=&\frac{1-r^{k+1}}{1-r}+\frac{(1-r)r^{k+1}}{1-r} && \text{Common denominator} \\
	=&\frac{1-r^{k+1}+r^{k+1}-r^{k+2}}{1-r} && \text{Expand the second term} \\
	=& \frac{1-r^{k+2}}{1-r} && \text{as required}
\end{align*}

This proves that the partial sums $S_k=\frac{1-r^{k+1}}{1-r}$. If we take the limit of the partial sums as $k\to\infty$, we can use the fact that since $-1<r<1$, $\limtoinf[k]r^{k+1}=0$. Therefore we have that

\begin{align*}
	\limtoinf[k]S_k=&\limtoinf[k]\frac{1-r^{k+1}}{1-r} \\
	=& \frac{1}{1-r}\cdot\limtoinf[k]\paren{1-r^{k+1}} && \text{Factor out scalar} \\
	=& \frac{1}{1-r}\paren{1- \limtoinf[k]r^{k+1}} && \text{Limit algebra} \\ 
	=& \frac{1}{1-r}\cdot\paren{1-0} && \text{Use }\limtoinf[k]r^{k+1}=0
\end{align*}

Thus showing that as the number of terms in the partial sums tends to $\infty$, we have that $\limtoinf[k]S_k=\series{n=0}{\infty}r^n=\frac{1}{1-r}$. Hence given $-1<r<1$, $\series{n=0}{\infty}r^n$ converges to $\frac{1}{1-r}$, as required.
\end{proof}

\newpage

% Begin Problem 2
\noindent \textbf{(2.5.3)} Decide the convergence or divergence of the following series.

\noindent (a) $\series{n=1}{\infty}\frac{3}{9n+1}$

\begin{proof}
	First, notice that $\frac{3}{9n+1}>0$ for all $n\in\bN$. We will compare this to $\frac{1}{n}$, which is also positive for all $n$. We can use the limit comparison test as follows
	
\begin{align*}
	\limtoinf\frac{\frac{3}{9n+1}}{\frac{1}{n}}=\limtoinf\frac{3n}{9n+1}=\limtoinf\frac{3}{9+\frac{1}{n}}=\frac{3}{9}
\end{align*}

Note that the last equality holds as $\limtoinf\frac{1}{n}=0$. Since we have used the limit comparison test to compare $\series{n=1}{\infty}\frac{1}{n}$ and $\series{n=1}{\infty}\frac{3}{9n+1}$, we have shown that $\series{n=1}{\infty}\frac{3}{9n+1}$ diverges as $\series{n=1}{\infty}\frac{1}{n}$ diverges and both series must either both converge or diverge if $0<\limtoinf\frac{a_n}{b_n}<\infty$.
\end{proof}


\noindent (b) $\series{n=1}{\infty}\frac{1}{2n-1}$

\begin{proof}
	Notice again that $\frac{1}{2n-1}>0$ for all $n\in\bN$. We will compare this to $\frac{1}{n}$, which is also positive for all $n$. We can use the limit comparison test as follows
	
\begin{align*}
	\limtoinf\frac{\frac{1}{2n-1}}{\frac{1}{n}}=\limtoinf\frac{n}{2n-1}=\limtoinf\frac{1}{2-\frac{1}{n}}=\frac{1}{2}
\end{align*}

Since we have used the limit comparison test to compare $\series{n=1}{\infty}\frac{1}{n}$ and $\series{n=1}{\infty}\frac{1}{2n-1}$, we have shown that the series diverges as $\series{n=1}{\infty}\frac{1}{n}$ diverges as $\frac{1}{2}$ is a positive constant.
\end{proof}

\noindent (c) $\series{n=1}{\infty}\frac{(-1)^n}{n^2}$

\begin{proof}
	We will prove this using the alternating series test, which is proved on \underline{J. Lebl pp.101}. The test states that if $\seq{x_n}$ is a monotone decreasing sequence of positive real number such that $\limtoinf x_n=0$, then $\series{n=1}{\infty}(-1)^nx_n$ converges. We must show, therefore, that given $x_n=\frac{1}{n^2},$ $\seq{x_n}$ is a monotone decreasing sequence where $x_n\ge 0$ for all $n\in\bN$ and that $\limtoinf\frac{1}{n^2}=0$. 
	
	Firstly, note that $\frac{1}{n^2}\ge0$ for all $n\in\bN$ as $n^2\ge0$. Next, we define $x_{n+1}=\frac{1}{(n+1)^2}$, and we know that $n^2\le(n+1)^2$ for all $n$ and hence $\frac{1}{n^2}\ge\frac{1}{(n+1)^2}$. Therefore $x_n\ge x_{n+1}$ so we have shown that $x_n$ is monotone decreasing and positive for all $n$. We can formally write that $x_n$ is monotone decreasing and bounded below by zero, so $\inf x_n = 0$, and hence the MCT tells us that $\limtoinf x_n=\limtoinf\frac{1}{n^2}=0$. Therefore as $\limtoinf\frac{1}{n^2}=0$ and $\frac{1}{n^2}$ is monotone decreasing, the series $\series{n=1}{\infty}\frac{(-1)^n}{n^2}$ converges.
\end{proof}

\noindent (d) $\series{n=1}{\infty}\frac{1}{n(n+1)}$

\begin{proof}
	First note that $\frac{1}{n}-\frac{1}{n+1}=\frac{1}{n(n+1)}$. This can be supported by
	
\begin{align*}
	\frac{1}{n}-\frac{1}{n+1}=\frac{(n+1)}{n(n+1)}-\frac{n}{n(n+1)}=\frac{(n+1)-n}{n(n+1)}=\frac{1}{n(n+1)}
\end{align*}

\noindent Hence we can express the series $\series{n=1}{\infty}\frac{1}{n(n+1)}$ as the telescoping sum

\begin{align*}
	\series{n=1}{\infty}\frac{1}{n(n+1)}=&\series{n=1}{\infty}\frac{1}{n}-\series{n=1}{\infty}\frac{1}{n+1} \\
	=&\paren{1-\frac{1}{2}}+\paren{\frac{1}{2}-\frac{1}{3}}+\paren{\frac{1}{3}-\frac{1}{4}}+\dots && \text{Write out the first few terms} \\
	=&1 && \text{All terms after 1 cancel out}
\end{align*}

	Therefore the series $\series{n=1}{\infty}\frac{1}{n(n+1)}$ converges as it is a telescoping series and more specifically converges to 1.

\end{proof}

\noindent (e) $\series{n=1}{\infty}ne^{-n^2}$

\begin{proof}
	We will choose to compare the series $\series{n=1}{\infty}\frac{n}{e^{n^2}}$ to $\series{n=1}{\infty}\frac{n}{n^3}$. First note that both $ne^{-n^2}$ and $\frac{n}{n^3}$ are positive for all $n\in\bN$. To justify this choice and , compare $e^{n^2}$ to $n^3$ (continuous functions) as
	
\begin{align*}
	\limtoinf\frac{e^{n^2}}{n^3}\overset{LH}{=}\limtoinf\frac{\frac{d}{dn}e^{n^2}}{\frac{d}{dn}n^3}=\limtoinf\frac{2ne^{n^2}}{3n}\to\infty \qquad \paren{\frac{\text{Exponential}}{\text{Linear}}\to\infty}.
\end{align*}

The limit of the terms tends to infinity showing that $e^{n^2}\ge n^3$ and hence $\frac{1}{e^{n^2}}\le\frac{1}{n^3}$. Consequently we have that $\series{n=1}{\infty}ne^{-n^2}\le\series{n=1}{\infty}\frac{n}{n^3}=\series{n=1}{\infty}\frac{1}{n^2},$ and $\series{n=1}{\infty}\frac{1}{n^2}$ converges by the p-series test with $p=2$. In summary we have shown that $0\le\series{n=1}{\infty}ne^{-n^2}\le\frac{1}{n^2}$. Therefore by the comparison test, $\series{n=1}{\infty}ne^{-n^2}$ converges.
\end{proof}

\newpage

% Begin Problem 3
\noindent \textbf{(2.5.14)} Suppose $\series{n=1}{\infty}x_n$ converges and $x_n\ge0$ for all $n$. Prove that $\series{n=1}{\infty} x_n^2$ converges.

\begin{proof}
	To determine the convergence of $\series{n=1}{\infty} x_n^2$, we can compare it to $\series{n=1}{\infty}x_n$. Because  $\series{n=1}{\infty}x_n$ is convergent, it implies the as $n$ gets sufficiently large, $x_n\to0$. In other words, the convergence of  $\series{n=1}{\infty}x_n$ implies that $\limtoinf x_n=0$. Therefore we have that $0<x<1$ for sufficiently large $n$. Notice that for $\series{n=1}{\infty} x_n^2$, we have $x_n^2\le x_n$ for sufficiently large $n$. Note that we have to prove this expression, and also that we only care about long term behavior as that is what ultimately determines convergence. We therefore have
	
\begin{align*}
	x^2\le&x&&\text{Want to show for }0<x<1 \\
	x - x^2\ge&0&&\text{Subtract }x\text{ from both sides} \\
	x(1-x)\ge&0&&\text{Factor the expression}
\end{align*} 

The following statement is true as $0<x<1$ is given and hence $x$ and $1-x$, the factors, are both always positive. Therefore $x^2\le x$ holds for all $0<x<1$.

Let $M\in\bN$ be given such that $n\ge M$, thus we are comparing the tails of the series when $0<x<1$ holds. We therefore have that $0\le x_n^2\le x_n$ for all $n\ge M$. Note that $\series{n=1}{\infty}x_n$ converges for $n\ge M$ as a series converges if and only if its tails converge. This also means that if a series tails converge, then the series also converges. Since $\series{n=1}{\infty}x_n$ converges and the terms in the long term bound the terms of $\sum x_n^2$, $\series{n=1}{\infty} x_n^2$  must also converge by the comparison test.
\end{proof}

\newpage

% Begin Problem 4
\noindent \textbf{(3.1.1)} Find the limit (and prove it of course) or prove that the limit does not exist.

\noindent (a) $\limit{x\to c} \sqrt{x}$, for $c\ge0$

\begin{proof}
	\lipsum[1][1-5]
\end{proof}

\noindent (b) $\limit{x\to c}x^2 + x + 1$, for $c\in\bR$

\begin{proof}
	\lipsum[1][1-5]
\end{proof}

\noindent (c) $\limit{x\to0}x^2\cos\paren{\frac{1}{x}}$

\begin{proof}
	\lipsum[1][1-5]
\end{proof}

\noindent (d) $\limit{x\to0}\sin\paren{\frac{1}{x}}\cos\paren{\frac{1}{x}}$

\begin{proof}
	\lipsum[1][1-5]
\end{proof}

\noindent (e) $\limit{x\to0}\sin\paren{0}\cos\paren{\frac{1}{x}}$

\begin{proof}
	\lipsum[1][1-5]
\end{proof}
\newpage

% Begin Problem 5
\noindent \textbf{(3.1.2)} Prove \underline{Corollary 3.1.10}, that is, let $S\subset\bR$ and let $c$ be a cluster point of $S$. Suppose $f:S\to\bR$ is a function such that the limit of $f(x)$ as $x$ goes to $c$ exists. Suppose there are two real numbers $a$ and $b$ such that

\begin{align*}
	a\le f(x)\le b \qquad \text{for all }x\in S\setminus\set{c}
\end{align*}
\noindent Then
\begin{align*}
	a\le\limit{x\to c}f(x)\le b.
\end{align*}

\begin{proof}
	\lipsum[1]
\end{proof}

\end{document}
