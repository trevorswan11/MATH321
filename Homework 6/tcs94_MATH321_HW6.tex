\documentclass[12pt]{article}

% Custom Commands
\newcommand{\set}[1]{\left\{ {#1} \right\}}
\newcommand{\limit}[1]{\displaystyle \lim_{ {#1} }}
\newcommand{\limtoinf}[1][n]{\lim_{ {#1} \to \infty}}
\newcommand{\liminftoinf}[1][n]{\liminf_{ {#1} \to \infty}}
\newcommand{\limsuptoinf}[1][n]{\limsup_{ {#1} \to \infty}}
\newcommand{\abs}[1]{\left| {#1} \right|}
\newcommand{\ceil}[1]{\lceil {#1} \rceil}
\newcommand{\floor}[1]{\lfloor {#1} \rfloor}
\newcommand{\seq}[2][n]{\left\{ {#2} \right\}_{#1=1}^\infty}
\newcommand{\paren}[1]{\left( {#1} \right)}
\newcommand{\series}[2]{\displaystyle \sum_{ {#1} }^{ {#2} }}
\newcommand{\bR}{\mathbb{R}}
\newcommand{\bZ}{\mathbb{Z}}
\newcommand{\bN}{\mathbb{N}}
\newcommand{\bQ}{\mathbb{Q}}

% Math and symbol packages
\usepackage{amsmath}
\usepackage{amsthm}
\usepackage{amssymb}
\usepackage{mathtools}
\usepackage{derivative}

% Formatting
\usepackage{inputenc}
\usepackage[left=2.54cm,right=2.54cm,top=2.54cm,bottom=2.54cm]{geometry}
\usepackage{fancyhdr}
\usepackage{lipsum}

% Spacing formats
\AtBeginDocument{
	\setlength{\abovedisplayskip}{-2pt}
	\setlength{\belowdisplayskip}{5pt}
}

% Actual content
\begin{document}
\pagestyle{fancy}
\setlength{\headheight}{14.49998pt}
\fancyhead[L]{Trevor Swan}
\fancyhead[C]{\textbf{MATH321 - HW6}}
\fancyhead[R]{10/11/24}
\fancyfoot[C]{\thepage}

% Begin Problem 1
\noindent \textbf{(2.5.2)} Prove \underline{Proposition 2.5.5}, that is, for $-1<r<1$, prove

\begin{align*}
	\series{n=0}{\infty}r^n=\frac{1}{1-r}
\end{align*}

\noindent given that the geometric series $\series{n=0}{\infty}r^n$ converges.

\begin{proof}
	\lipsum[1]
\end{proof}

\newpage

% Begin Problem 2
\noindent \textbf{(2.5.3)} Decide the convergence or divergence of the following series.

\noindent (a) $\series{n=1}{\infty}\frac{3}{9n+1}$

\begin{proof}
	\lipsum[1][1-5]
\end{proof}

\noindent (b) $\series{n=1}{\infty}\frac{1}{2n-1}$

\begin{proof}
	\lipsum[1][1-5]
\end{proof}

\noindent (c) $\series{n=1}{\infty}\frac{(-1)^n}{n^2}$

\begin{proof}
	\lipsum[1][1-5]
\end{proof}

\noindent (d) $\series{n=1}{\infty}\frac{1}{n(n+1)}$

\begin{proof}
	\lipsum[1][1-5]
\end{proof}

\noindent (e) $\series{n=1}{\infty}ne^{-n^2}$

\begin{proof}
	\lipsum[1][1-5]
\end{proof}

\newpage

% Begin Problem 3
\noindent \textbf{(2.5.14)} Suppose $\series{n=1}{\infty}x_n$ converges and $x_n\ge0$ for all $n$. Prove that $\series{n=1}{\infty} x_n^2$ converges.

\begin{proof}
	\lipsum[1]
\end{proof}

\newpage

% Begin Problem 4
\noindent \textbf{(3.1.1)} Find the limit (and prove it of course) or prove that the limit does not exist.

\noindent (a) $\limit{x\to c} \sqrt{x}$, for $c\ge0$

\begin{proof}
	\lipsum[1][1-5]
\end{proof}

\noindent (b) $\limit{x\to c}x^2 + x + 1$, for $c\in\bR$

\begin{proof}
	\lipsum[1][1-5]
\end{proof}

\noindent (c) $\limit{x\to0}x^2\cos\paren{\frac{1}{x}}$

\begin{proof}
	\lipsum[1][1-5]
\end{proof}

\noindent (d) $\limit{x\to0}\sin\paren{\frac{1}{x}}\cos\paren{\frac{1}{x}}$

\begin{proof}
	\lipsum[1][1-5]
\end{proof}

\noindent (e) $\limit{x\to0}\sin\paren{0}\cos\paren{\frac{1}{x}}$

\begin{proof}
	\lipsum[1][1-5]
\end{proof}
\newpage

% Begin Problem 5
\noindent \textbf{(3.1.2)} Prove \underline{Corollary 3.1.10}, that is, let $S\subset\bR$ and let $c$ be a cluster point of $S$. Suppose $f:S\to\bR$ is a function such that the limit of $f(x)$ as $x$ goes to $c$ exists. Suppose there are two real numbers $a$ and $b$ such that

\begin{align*}
	a\le f(x)\le b \qquad \text{for all }x\in S\setminus\set{c}
\end{align*}
\noindent Then
\begin{align*}
	a\le\limit{x\to c}f(x)\le b.
\end{align*}

\begin{proof}
	\lipsum[1]
\end{proof}

\end{document}
