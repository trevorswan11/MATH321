\documentclass[12pt]{article}

% Custom Commands
\newcommand{\set}[1]{\left\{ {#1} \right\}}
\newcommand{\limit}[1]{\displaystyle \lim_{ {#1} }}
\newcommand{\limtoinf}[1][n]{\displaystyle\lim_{ {#1} \to \infty}}
\newcommand{\liminftoinf}[1][n]{\displaystyle\liminf_{ {#1} \to \infty}}
\newcommand{\limsuptoinf}[1][n]{\displaystyle\limsup_{ {#1} \to \infty}}
\newcommand{\abs}[1]{\left| {#1} \right|}
\newcommand{\ceil}[1]{\lceil {#1} \rceil}
\newcommand{\floor}[1]{\lfloor {#1} \rfloor}
\newcommand{\seq}[2][n]{\left\{ {#2} \right\}_{#1=1}^\infty}
\newcommand{\paren}[1]{\left( {#1} \right)}
\newcommand{\series}[2]{\displaystyle \sum_{ {#1} }^{ {#2} }}
\newcommand{\bR}{\mathbb{R}}
\newcommand{\bZ}{\mathbb{Z}}
\newcommand{\bN}{\mathbb{N}}
\newcommand{\bQ}{\mathbb{Q}}

% Math and symbol packages
\usepackage{amsmath}
\usepackage{amsthm}
\usepackage{amssymb}
\usepackage{mathtools}
\usepackage{derivative}

% Formatting
\usepackage{inputenc}
\usepackage[left=2.54cm,right=2.54cm,top=2.54cm,bottom=2.54cm]{geometry}
\usepackage{fancyhdr}
\usepackage{lipsum}

% Spacing formats
\AtBeginDocument{
	\setlength{\abovedisplayskip}{-2pt}
	\setlength{\belowdisplayskip}{5pt}
}

% Actual content
\begin{document}
\pagestyle{fancy}
\setlength{\headheight}{14.49998pt}
\fancyhead[L]{Trevor Swan}
\fancyhead[C]{\textbf{MATH321 - HW7}}
\fancyhead[R]{10/23/24}
\fancyfoot[C]{\thepage}

% Begin Problem 1
\noindent \textbf{(3.1.5)} Let $A\subset S$. Show that if $c$ is a cluster point of $A$, then $c$ is a cluster point of $S$. \textit{Note the difference from \underline{Proposition 3.1.15}}

\begin{proof}
	A point $c$ is a cluster point of a set $A$ if every neighborhood or $c$ contains a point of $A$ that is distinct from $c$. Similarly, $c$ is a cluster point is a cluster point of $S$ if every neighborhood of $c$ contains a point of $S$ distinct from $c$. Assume that $c$ is a cluster point of $A$. This means that for every $\epsilon>0$, there is some $x\in A$ with $x\neq c$ such that $\abs{x-c}<\epsilon$. More specifically it means that $x\in\paren{c-\epsilon, c+\epsilon}\cap [S\setminus\set{c}]$. Since $A\subset S$, every point $a\in A$ must also be a point in $S$ by definition. Therefore, for every $x\in\paren{c-\epsilon, c+\epsilon}\cap [S\setminus\set{c}]$, we know that $x\in A\subset S$. This implies that every $\epsilon$ yields an $x$ abiding by the previous description such that $x\in S\setminus\set{c}$, meaning that $c$ must be a cluster point of $S$.
\end{proof}

\newpage

% Begin Problem 2
\noindent \textbf{(3.1.12)} Prove \underline{Proposition 3.1.17}, that is, Let $S\subset\bR$ be such that $c$ is a cluster point of both $S\cap\paren{-\infty, c}$ and $S\cap\paren{c,\infty}$, let $f:S\to\bR$ be a function, and let $L\in\bR$. Then $c$ is a cluster point of $S$ and

\begin{align*}
	\limit{x\to c}f(c)=L\qquad\text{if and only if}\qquad\limit{x\to c^-}f(x)=\limit{x\to c^+}f(x)=L
\end{align*}

\begin{proof}
	We begin with the forward direction. Assume that $\limit{x\to c}f(x)=L$. By the definition of the limit, for every $\epsilon>0$, there exists some $\delta>0$ such that $\abs{f(x)-L}<\epsilon$ whenever $0<\abs{x-c}<\delta$. We can split this into two parts:
	
\begin{enumerate}
	\item For the left-hand limit, $x\to c^-$: If $x<c$ and $0<c-x<\delta$, then $\abs{f(x)-L}<\epsilon$. Thus, $\limit{x\to c^-}f(x)=L$.
	\item For the right-hand limit,  $x\to c^+$: If $x>c$ and $0<x-c<\delta$, then $\abs{f(x)-L}<\epsilon$. Thus, $\limit{x\to c^+}f(x)=L$.
\end{enumerate}

\noindent Since both limits equal $L$, we conclude that $\limit{x\to c^-}f(x)=\limit{x\to c^+}f(x)=L$.

\indent Now we must prove the reverse direction. Suppose that $\limit{x\to c^-}f(x)=\limit{x\to c^+}f(x)=L$. By the definition of left- and right-hand limits, we have:

\begin{enumerate}
	\item For the left-hand limit, given $\epsilon>0$, there exists some $\delta_1>0$ such that $\abs{f(x)-L}<\epsilon$ whenever $0<c-x<\delta_1$.
	\item For the right-hand limit, given the same arbitrarily chosen $\epsilon>0$, there exists some $\delta_2>0$ such that $\abs{f(x)-L}<\epsilon$ whenever $0<x-c<\delta_1$.
\end{enumerate}

\indent We must connect $\delta_1$ and $\delta_2$. Let $\delta = \min\paren{\delta_1,\delta_2}$. This ensures that for all $x$ sufficiently close to $c$, the function values $f(x)$ are close to $L$ as this $x$ satisfies $0<\abs{x-c}<\delta$ (as it was said to be sufficiently close). Here, $x$ will either fall into the interval $\paren{c-\delta, c}$ (for $x<c$) or $\paren{c,c+\delta}$ (for $x>c$). Therefore for such $x$ we can say $\abs{f(x)-L}<\epsilon$. By definition this shows that $\limit{x\to c}f(c)=L$, as required.
\end{proof}

\newpage

% Begin Problem 3
\noindent \textbf{(3.2.2)} Using the definition of continuity directly prove that $f:\paren{0,\infty}\to\bR$ defined by $f(x)\coloneq\frac{1}{x}$ is continuous.

\begin{proof}
	To directly prove that the function $f:\paren{0,\infty}\to\bR$ given by $f(x)\coloneq\frac{1}{x}$ is continuous, we must show that at any point $c\in\paren{0,\infty}$, the epsilon-delta definition of continuity holds. That is, we need to show that for every $\epsilon>0$, there exists some $\delta>0$ such that if $0<\abs{x-c}<\delta$, then $\abs{f(x)-f(c)}<\epsilon$.

\indent We begin by computing $\abs{f(x)-f(c)}=\abs{\frac{1}{x}-\frac{1}{c}}$. We can simplify this by finding a common denominator as $\abs{\frac{1}{x}-\frac{1}{c}}=\abs{\frac{c-x}{xc}}=\frac{\abs{c-x}}{\abs{xc}}$. Now we must show that this expression is lees than $\epsilon$, or that $\frac{\abs{c-x}}{\abs{xc}}<\epsilon$. We can rewrite the previous inequality as $\abs{c-x}<\epsilon\abs{xc}$.

\indent We must now find $\delta$, which requires controlling both $\abs{x-c}$ and $\abs{xc}$. As $c>0$, set $\abs{x-c}<\delta$. We can expand this to be $-\delta<x-c<\delta\implies c-\delta<x<c+\delta$. If we take $\delta\le\frac{c}{2}$, then this ensures that $x$ remains close to $c$. Thus we have $c-\frac{c}{2}<x<c+\frac{c}{2}\implies\frac{c}{2}<c<\frac{3c}{2}$. As this $x$ must be greater than $\frac{c}{2}$, we can bound $\abs{xc}$ as $\abs{xc}>\paren{\frac{c}{2}c}=\frac{c^2}{2}$. We can substitute these back into $\abs{c-x}<\epsilon\abs{xc}$ to yield $\abs{c-x}<\epsilon\frac{c^2}{2}$. This allows us to choose $\delta=\min\paren{\frac{c}{2}, \frac{\epsilon c^2}{2}}$.

\indent We can now proceed by verifying this choice of $\delta$. We know from above that if $\abs{x-c}<\delta$, then $\abs{c-x}<\frac{\epsilon c^2}{2}$. We also have that $\abs{xc}>\frac{c^2}{2}$,  so

\begin{align*}
	\abs{f(x)-f(c)}=\frac{\abs{c-x}}{\abs{xc}}<\frac{\frac{\epsilon c^2}{2}}{\frac{c^2}{2}}=\epsilon. 
\end{align*}

\noindent and hence $f(x)\coloneq\frac{1}{x}$ is continuous at every $c\in\paren{0,\infty}$, as required.
\end{proof}

\newpage

% Begin Problem 4
\noindent \textbf{(3.2.13)} Let $f:S\to\bR$ be a function and $c\in S$, such that for every sequence $\seq{x_n}$ in $S$ with $\limtoinf x_n=c$, the sequence $\seq{f(x_n)}$ converges. Show that $f$ is continuous at $c$.

\begin{proof}
	We know that $f$ satisfies the property that for every sequence $\seq{x_n}$ in $S$ with $\limtoinf x_n=c$, the sequence $\seq{f(x_n)}$ converges. We must show continuity at $c$, meaning that $\limit{x\to c}f(x)=f(c)$.
	
\indent Assume for contradiction that $f$ is not continuous at $c$. By definition of continuity, it must consequently be the case that $\limit{x\to c}f(x)\neq f(c)$. This implies that there exists some sequence $\seq{x_n}$ in $S$ such that $\limtoinf x_n=c$, but that the corresponding sequence $\seq{f(x_n)}$ does not converge to $f(c)$. However, we have assumed that for every sequence $\seq{x_n}$ with $\limtoinf x_n=c$, the corresponding sequence $\seq{f(x_n)}$ does, in fact, converge. Thus we arrive at a contradiction indicating that our assumption of $f$ not being continuous is false. Therefore $f$ is continuous at $c$, as required.
\end{proof}

\newpage

% Begin Problem 5
\noindent \textbf{(3.2.15)} Suppose $g:\bR\to\bR$ is a continuous function such that $g(0)=0$, and suppose $f:\bR\to\bR$ is such that $\abs{f(x)-f(y)}\le g(x-y)$ for all $x$ and $y$. Show that $f$ is continuous.

\begin{proof}
	We are given that $\abs{f(x)-f(y)}\le g(x-y)$, and letting $y=c$, we have $\abs{f(x)-f(c)}\le g(x-c)$. We can do this as this inequality is given to hold true for all $x,y\in\bR$, and we know $c\in\bR$. We proceed by analyzing the behavior of $g$ as $x\to c$. In other words, we must see the behavior of $\limit{x\to c}g(x-c)$. Since $x\to c$ implies that $x-c\to 0$, we have that $\limit{x\to c}g(x-c)=g(0)=0$.

\indent We will proceed with the use of the squeeze theorem. Since $\abs{f(x)-f(c)}\le g(x-c)$, we have that $-g(x-c)\le f(x)-f(c)\le g(x-c)$. We take the limit of this expression as:

\begin{align*}
	\limit{x\to c}\paren{-g(x-c)\le f(x)-f(c)\le g(x-c)}=-\limit{x\to c}g(x-c)\le\limit{x\to c}f(x)-f(c)\le\limit{x\to c}g(x-c).
\end{align*}

\noindent Therefore we have that $\limit{x\to c}f(x)-f(c)=0$ by the squeeze theorem as $\limit{x\to c}g(x-c)=g(0)=0$. We thus conclude that $\limit{x\to c}\abs{f(x)-f(c)}=0$. As $f(c)$ is not dependent on $x$, we can rewrite this expression as $\limit{x\to c}\abs{f(x)}=\abs{f(c)}$. This shows that $f$ is continuous at $c$, and since $c$ was chosen arbitrarily, we therefore have that $f$ is continuous everywhere on $\bR$.
\end{proof}

\end{document}
