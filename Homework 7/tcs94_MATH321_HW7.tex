\documentclass[12pt]{article}

% Custom Commands
\newcommand{\set}[1]{\left\{ {#1} \right\}}
\newcommand{\limit}[1]{\displaystyle \lim_{ {#1} }}
\newcommand{\limtoinf}[1][n]{\displaystyle\lim_{ {#1} \to \infty}}
\newcommand{\liminftoinf}[1][n]{\displaystyle\liminf_{ {#1} \to \infty}}
\newcommand{\limsuptoinf}[1][n]{\displaystyle\limsup_{ {#1} \to \infty}}
\newcommand{\abs}[1]{\left| {#1} \right|}
\newcommand{\ceil}[1]{\lceil {#1} \rceil}
\newcommand{\floor}[1]{\lfloor {#1} \rfloor}
\newcommand{\seq}[2][n]{\left\{ {#2} \right\}_{#1=1}^\infty}
\newcommand{\paren}[1]{\left( {#1} \right)}
\newcommand{\series}[2]{\displaystyle \sum_{ {#1} }^{ {#2} }}
\newcommand{\bR}{\mathbb{R}}
\newcommand{\bZ}{\mathbb{Z}}
\newcommand{\bN}{\mathbb{N}}
\newcommand{\bQ}{\mathbb{Q}}

% Math and symbol packages
\usepackage{amsmath}
\usepackage{amsthm}
\usepackage{amssymb}
\usepackage{mathtools}
\usepackage{derivative}

% Formatting
\usepackage{inputenc}
\usepackage[left=2.54cm,right=2.54cm,top=2.54cm,bottom=2.54cm]{geometry}
\usepackage{fancyhdr}
\usepackage{lipsum}

% Spacing formats
\AtBeginDocument{
	\setlength{\abovedisplayskip}{-2pt}
	\setlength{\belowdisplayskip}{5pt}
}

% Actual content
\begin{document}
\pagestyle{fancy}
\setlength{\headheight}{14.49998pt}
\fancyhead[L]{Trevor Swan}
\fancyhead[C]{\textbf{MATH321 - HW7}}
\fancyhead[R]{10/23/24}
\fancyfoot[C]{\thepage}

% Begin Problem 1
\noindent \textbf{(3.1.5)} Let $A\subset S$. Show that if $c$ is a cluster point of $A$, then $c$ is a cluster point of $S$. \textit{Note the difference from \underline{Proposition 3.1.15}}

\begin{proof}
	A point $c$ is a cluster point of a set $A$ if every neighborhood or $c$ contains a point of $A$ that is distinct from $c$. Similarly, $c$ is a cluster point is a cluster point of $S$ if every neighborhood of $c$ contains a point of $S$ distinct from $c$. Assume that $c$ is a cluster point of $A$. This means that for every $\epsilon>0$, there is some $x\in A$ with $x\neq c$ such that $\abs{x-c}<\epsilon$. More specifically it means that $x\in\paren{c-\epsilon, c+\epsilon}\cap [S\setminus\set{c}]$. Since $A\subset S$, every point $a\in A$ must also be a point in $S$ by definition. Therefore, for every $x\in\paren{c-\epsilon, c+\epsilon}\cap [S\setminus\set{c}]$, we know that $x\in A\subset S$. This implies that every $\epsilon$ yields an $x$ abiding by the previous description such that $x\in S\setminus\set{c}$, meaning that $c$ must be a cluster point of $S$.
\end{proof}

\newpage

% Begin Problem 2
\noindent \textbf{(3.1.12)} Prove \underline{Proposition 3.1.17}, that is, Let $S\subset\bR$ be such that $c$ is a cluster point of both $S\cap\paren{-\infty, c}$ and $S\cap\paren{c,\infty}$, let $f:S\to\bR$ be a function, and let $L\in\bR$. Then $c$ is a cluster point of $S$ and

\begin{align*}
	\limit{x\to c}f(c)=L\qquad\text{if and only if}\qquad\limit{x\to c^-}f(x)=\limit{x\to c^+}f(x)=L
\end{align*}

\begin{proof}
	\lipsum[1]
\end{proof}

\newpage

% Begin Problem 3
\noindent \textbf{(3.2.2)} Using the definition of continuity directly prove that $f:\paren{0,\infty}\to\bR$ defined by $f(x)\coloneq\frac{1}{x}$ is continuous.

\begin{proof}
	\lipsum[1]
\end{proof}

\newpage

% Begin Problem 4
\noindent \textbf{(3.2.13)} Let $f:S\to\bR$ be a function and $c\in S$, such that for every sequence $\seq{x_n}$ in $S$ with $\limtoinf x_n=c$, the sequence $\seq{f(x_n)}$ converges. Show that $f$ is continuous at $c$.

\begin{proof}
	\lipsum[1]
\end{proof}

\newpage

% Begin Problem 5
\noindent \textbf{(3.2.15)} Suppose $g:\bR\to\bR$ is a continuous function such that $g(0)=0$, and suppose $f:\bR\to\bR$ is such that $\abs{f(x)-f(y)}\le g(x-y)$ for all $x$ and $y$. Show that $f$ is continuous.

\begin{proof}
	\lipsum[1]
\end{proof}

\end{document}
